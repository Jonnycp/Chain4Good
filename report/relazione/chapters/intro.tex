% !TeX spellcheck = it_IT
% !TEX TS-program = pdflatex
% !TEX root = ../main.tex


% ********************************************************************
\section{Introduzione}
\label{sec:introduzione}
% ********************************************************************

Il \gls{cf} rappresenta uno strumento di finanziamento collettivo, capace di aggregare un elevato numero di contributi economici di modesta entità a supporto di progetti e iniziative di diversa natura. In particolare, nel contesto del Terzo Settore, il \gls{cf} rappresenta un mezzo efficace per sostenere attività a fini sociali, culturali e umanitari, consentendo agli enti beneficiari di raggiungere una platea ampia e diversificata di donatori attraverso piattaforme digitali.

Nonostante la crescente diffusione, le piattaforme di \gls{cf} tradizionali presentano criticità strutturali riconducibili principalmente alla loro architettura centralizzata. 
In particolare, la gestione fiduciaria dei fondi raccolti e la limitata trasparenza del loro utilizzo rappresentano fattori capaci di incidere negativamente sul livello di fiducia degli utenti, tale da compromettere la loro partecipazione ai processi di donazione. 

In questo scenario, la blockchain si configura come una soluzione promettente per il superamento di questi limiti. Grazie alle sue proprietà di decentralizzazione, immutabilità e trasparenza, la blockchain ha il potenziale di ridefinire il paradigma delle piattaforme attuali, il cui corretto funzionamento risulta fortemente vincolato dal comportamento onesto degli intermediari e dei promotori delle iniziative.

Alla luce di queste considerazioni, il presente lavoro propone \textit{Chain4Good}, una piattaforma di \gls{cf} decentralizzata, nata specificamente per supportare iniziative promosse dagli Enti del Terzo settore.
Il suo obiettivo principale è restituire al donatore un ruolo attivo lungo l’intero ciclo di vita delle donazioni. Per farlo, implementa un meccanismo di erogazione incrementale dei fondi, basato su processo di approvazione decentralizzato, attraverso il quale i donatori sono chiamati a votare sulle singole richieste di spesa presentate dai beneficiari. In questo senso, la piattaforma proposta ha il potenziale di ridefinire il concetto di donazione trasformandolo da un mero gesto di fiducia in un processo intrinsecamente sicuro, trasparente e verificabile in ogni sua fase.



%Diversi contributi scientifici hanno proposto l’adozione della tecnologia blockchain come meccanismo per migliorare la tracciabilità, l’immutabilità e l’auditabilità delle donazioni. Tuttavia, una parte significativa delle soluzioni esistenti si concentra principalmente sulla registrazione delle transazioni o sul rilascio automatico dei fondi al raggiungimento di obiettivi prefissati, trascurando il coinvolgimento attivo dei donatori nelle fasi successive alla raccolta. Ne deriva che, sebbene la blockchain aumenti la trasparenza tecnica delle operazioni, il controllo effettivo sull’uso delle risorse rimane spesso limitato.
%In questo contesto si colloca Chain4Good, una piattaforma di crowdfunding benefico decentralizzato che mira a estendere i principi di trasparenza e responsabilità lungo l’intero ciclo di vita del progetto finanziato. In linea con le proposte presenti in letteratura, Chain4Good sfrutta la blockchain e gli smart contract per eliminare intermediari centralizzati e garantire l’immutabilità delle transazioni. Tuttavia, il sistema introduce un ulteriore livello di controllo attraverso la vincolazione crittografica dei fondi a specifici progetti, identificati univocamente mediante un ProjectID.

% L’impiego di tale tecnologia nel \gls{cf}, dunque, risponde a specifiche esigenze di affidabilità, sicurezza e tracciabilità delle donazioni.



%Risulta evidente che il crowdfunding rappresenti uno strumento fondamentale per la democratizzazione dell’accessibilità ai finanziamenti per una vasta gamma di progetti e organizzazioni; tuttavia, i limiti caratteristici delle piattaforme tradizionali hanno evidenziato la necessità di un’evoluzione verso modelli Web3 – Kickstarter stessa ne è ha preso atto in un comunicato del 2021
