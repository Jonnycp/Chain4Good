% !TeX spellcheck = it_IT
% !TEX TS-program = pdflatex
% !TEX root = ../main.tex


% ********************************************************************
\section{Introduzione}
\label{sec:introduzione}
% ********************************************************************


%Nata come infrastruttura di supporto per la criptovaluta Bitcoin, la blockchain si è progressivamente evoluta in una tecnologia \textit{general-purpose}, trovando applicazione in un'ampia gamma di contesti. \cite{panarello2018blockchain}.

%Il crowdfunding è un modello di finanziamento collettivo in cui un promotore presenta un progetto e raccoglie contributi economici da un insieme di sostenitori tramite una piattaforma digitale. Nelle soluzioni tradizionali, tali piattaforme operano come intermediari centralizzati, occupandosi della gestione delle campagne, dei pagamenti e, in molti casi, della distribuzione dei fondi raccolti.
%In questo contesto, l’impiego della tecnologia blockchain è stato proposto come soluzione per migliorare trasparenza, affidabilità e responsabilità nei sistemi di \textit{crowdfunding}. Grazie alle proprietà di immutabilità e verificabilità del registro distribuito, la blockchain consente di tracciare in modo pubblico e non alterabile le transazioni associate a una campagna di finanziamento. L’utilizzo di \textit{Smart Contract} permette inoltre di automatizzare le regole di gestione dei fondi, vincolando l’erogazione a condizioni prestabilite, come il raggiungimento di una soglia di finanziamento o l’approvazione da parte dei donatori.
%Lo stato dell’arte mostra una crescente attenzione verso piattaforme di crowdfunding decentralizzate, nelle quali la blockchain viene utilizzata come livello di fiducia e controllo, mentre componenti off-chain sono impiegate per garantire scalabilità ed efficienza nella gestione dei dati. Le soluzioni proposte mirano in particolare a ridurre la dipendenza da intermediari, incrementare la trasparenza post-finanziamento e offrire ai sostenitori un maggiore controllo sul destino dei fondi versati.
