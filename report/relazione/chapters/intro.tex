% !TeX spellcheck = it_IT
% !TEX TS-program = pdflatex
% !TEX root = ../main.tex


% ********************************************************************
\section{Introduzione}
\label{sec:introduzione}
% ********************************************************************

Negli ultimi anni, il \gls{cf} si è affermato come uno strumento innovativo di finanziamento collettivo, capace di aggregare un elevato numero di contributi economici di modesta entità a supporto di progetti e iniziative di diversa natura. In particolare, nel contesto del Terzo Settore, il crowdfunding rappresenta un mezzo efficace per sostenere attività a fini sociali, culturali e umanitari, consentendo agli enti beneficiari di raggiungere una platea ampia e diversificata di donatori attraverso piattaforme digitali.

Nonostante la crescente diffusione, le piattaforme di \gls{cf} tradizionali presentano criticità strutturali riconducibili principalmente alla loro architettura centralizzata. In tali sistemi, la gestione dei fondi e delle informazioni è demandata a intermediari fiduciari, limitando la trasparenza sull’utilizzo delle risorse raccolte e riducendo il controllo effettivo dei donatori sull’intero ciclo di vita della donazione. Questo aspetto risulta particolarmente critico nel crowdfunding filantropico, dove la fiducia e la tracciabilità rappresentano elementi essenziali per garantire la credibilità delle iniziative e incentivare la partecipazione degli utenti.

In questo scenario, la tecnologia \textit{blockchain} si configura come una soluzione promettente per il superamento di tali limiti. Grazie alle sue proprietà intrinseche di decentralizzazione, immutabilità, trasparenza e sicurezza, la blockchain consente di realizzare sistemi in cui le operazioni risultano verificabili e resistenti a manomissioni, riducendo la necessità di intermediari centrali. L’impiego degli \textit{Smart Contract}, inoltre, permette di automatizzare la gestione delle donazioni e dei flussi finanziari sulla base di regole predefinite e pubblicamente verificabili.

Tuttavia, come evidenziato dalla letteratura, l’adozione di modelli completamente decentralizzati introduce nuove sfide, sia dal punto di vista progettuale che operativo. In particolare, l’assenza di entità di supervisione richiede l’implementazione di meccanismi di tutela aggiuntivi e può comportare un aumento della complessità per gli utenti, soprattutto in contesti caratterizzati da attori con competenze tecnologiche eterogenee. Nel dominio del crowdfunding filantropico, risulta quindi necessario bilanciare i benefici della decentralizzazione con esigenze di affidabilità, semplicità d’uso e responsabilità.

Alla luce di queste considerazioni, il presente lavoro propone \textit{Chain4Good}, una piattaforma di crowdfunding basata su tecnologia blockchain, progettata specificamente per il Terzo Settore. L’obiettivo principale è restituire al donatore un ruolo attivo e continuo nel processo di donazione, garantendo al contempo trasparenza e tracciabilità nell’utilizzo dei fondi. A differenza dei modelli tradizionali, Chain4Good adotta un approccio ibrido che combina meccanismi di verifica off-chain sugli enti promotori con una governance decentralizzata on-chain, basata su \textit{Smart Contract} e processi di approvazione partecipativa delle richieste di spesa.











%Diversi contributi scientifici hanno proposto l’adozione della tecnologia blockchain come meccanismo per migliorare la tracciabilità, l’immutabilità e l’auditabilità delle donazioni. Tuttavia, una parte significativa delle soluzioni esistenti si concentra principalmente sulla registrazione delle transazioni o sul rilascio automatico dei fondi al raggiungimento di obiettivi prefissati, trascurando il coinvolgimento attivo dei donatori nelle fasi successive alla raccolta. Ne deriva che, sebbene la blockchain aumenti la trasparenza tecnica delle operazioni, il controllo effettivo sull’uso delle risorse rimane spesso limitato.
%In questo contesto si colloca Chain4Good, una piattaforma di crowdfunding benefico decentralizzato che mira a estendere i principi di trasparenza e responsabilità lungo l’intero ciclo di vita del progetto finanziato. In linea con le proposte presenti in letteratura, Chain4Good sfrutta la blockchain e gli smart contract per eliminare intermediari centralizzati e garantire l’immutabilità delle transazioni. Tuttavia, il sistema introduce un ulteriore livello di controllo attraverso la vincolazione crittografica dei fondi a specifici progetti, identificati univocamente mediante un ProjectID.

% L’impiego di tale tecnologia nel \gls{cf}, dunque, risponde a specifiche esigenze di affidabilità, sicurezza e tracciabilità delle donazioni.



%Risulta evidente che il crowdfunding rappresenti uno strumento fondamentale per la democratizzazione dell’accessibilità ai finanziamenti per una vasta gamma di progetti e organizzazioni; tuttavia, i limiti caratteristici delle piattaforme tradizionali hanno evidenziato la necessità di un’evoluzione verso modelli Web3 – Kickstarter stessa ne è ha preso atto in un comunicato del 2021
