% !TeX spellcheck = it_IT
% !TEX TS-program = pdflatex
% !TEX root = ../main.tex


% ********************************************************************
\section{Introduzione}
\label{sec:introduzione}
% ********************************************************************

%Diversi contributi scientifici hanno proposto l’adozione della tecnologia blockchain come meccanismo per migliorare la tracciabilità, l’immutabilità e l’auditabilità delle donazioni. Tuttavia, una parte significativa delle soluzioni esistenti si concentra principalmente sulla registrazione delle transazioni o sul rilascio automatico dei fondi al raggiungimento di obiettivi prefissati, trascurando il coinvolgimento attivo dei donatori nelle fasi successive alla raccolta. Ne deriva che, sebbene la blockchain aumenti la trasparenza tecnica delle operazioni, il controllo effettivo sull’uso delle risorse rimane spesso limitato.
%In questo contesto si colloca Chain4Good, una piattaforma di crowdfunding benefico decentralizzato che mira a estendere i principi di trasparenza e responsabilità lungo l’intero ciclo di vita del progetto finanziato. In linea con le proposte presenti in letteratura, Chain4Good sfrutta la blockchain e gli smart contract per eliminare intermediari centralizzati e garantire l’immutabilità delle transazioni. Tuttavia, il sistema introduce un ulteriore livello di controllo attraverso la vincolazione crittografica dei fondi a specifici progetti, identificati univocamente mediante un ProjectID.

% L’impiego di tale tecnologia nel \gls{cf}, dunque, risponde a specifiche esigenze di affidabilità, sicurezza e tracciabilità delle donazioni.



%Risulta evidente che il crowdfunding rappresenti uno strumento fondamentale per la democratizzazione dell’accessibilità ai finanziamenti per una vasta gamma di progetti e organizzazioni; tuttavia, i limiti caratteristici delle piattaforme tradizionali hanno evidenziato la necessità di un’evoluzione verso modelli Web3 – Kickstarter stessa ne è ha preso atto in un comunicato del 2021
