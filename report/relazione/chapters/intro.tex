% !TeX spellcheck = it_IT
% !TEX TS-program = pdflatex
% !TEX root = ../main.tex


% ********************************************************************
\section{Introduzione}
\label{sec:introduzione}
% ********************************************************************

Il \gls{cf} rappresenta uno strumento di finanziamento collettivo, capace di aggregare un elevato numero di contributi economici di modesta entità a supporto di progetti e iniziative di diversa natura. In particolare, nel contesto del Terzo Settore, il \gls{cf} rappresenta un mezzo efficace per sostenere attività a fini sociali, culturali e umanitari, consentendo agli enti beneficiari di raggiungere una platea ampia e diversificata di donatori attraverso piattaforme digitali.

Nonostante la sua crescente diffusione, i sistemi tradizionali presentano criticità strutturali riconducibili principalmente alla loro architettura centralizzata. 
In particolare, la gestione fiduciaria dei fondi raccolti e la limitata trasparenza del loro utilizzo rappresentano fattori capaci di incidere negativamente sul livello di fiducia degli utenti, tale da compromettere la loro partecipazione ai processi di donazione. 

In questo scenario, la blockchain si configura come una soluzione promettente per il superamento di questi limiti. Le sue proprietà di decentralizzazione, immutabilità e trasparenza permettono infatti di ridurre la dipendenza da intermediari fiduciari, ridefinendo il paradigma delle piattaforme di \gls{cf} tradizionali.

Alla luce di queste considerazioni, il presente lavoro propone \textit{Chain4Good}, una piattaforma di \gls{cf} decentralizzata, nata specificamente per supportare iniziative promosse dagli Enti del Terzo settore. 
Il suo obiettivo principale è restituire al donatore un ruolo attivo lungo l’intero ciclo di vita delle donazioni. Per farlo, implementa un meccanismo di erogazione incrementale dei fondi, basato su processo di approvazione decentralizzato, attraverso il quale i donatori sono chiamati a votare sulle singole richieste di spesa presentate dai beneficiari. In questo senso, la piattaforma proposta ha il potenziale di ridefinire il concetto di donazione trasformandolo da un mero gesto di fiducia in un processo intrinsecamente sicuro, trasparente e verificabile in ogni sua fase.
