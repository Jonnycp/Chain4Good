% !TeX spellcheck = it_IT
% !TEX TS-program = pdflatex
% !TEX root = ../main.tex


% ********************************************************************
\section{Background}
\label{sec:background}
% ********************************************************************

\subsection{Blockchain}
Una blockchain è una base di dati distribuita (\textit{distributed ledger}), condivisa e immutabile, gestita da una rete di nodi interconnessi secondo un’architettura \gls{p2p}. \\
Tale paradigma architetturale è stato introdotto per la prima volta nel 2008 con la pubblicazione del \textit{white paper} “Bitcoin: A Peer-to-Peer Electronic Cash System” da parte di Satoshi Nakamoto, con l'obiettivo di creare un sistema di pagamento elettronico decentralizzato che permettesse a due controparti di effettuare transazioni in modo sicuro, senza l'ausilio di autorità centrali \cite{nakamoto2007bitcoin}. \\
Sul piano architetturale, la blockchain è strutturata come una catena sequenziale di blocchi, ciascuno contenente un insieme di transazioni validate dalla rete. 

\subsubsection{Transazioni}
Una transazione costituisce l’unità fondamentale di informazione all’interno della blockchain. Essa descrive un’operazione richiesta da un utente, come il trasferimento di un asset digitale (\textit{token}) \cite{crosby2016blockchain}.
Ogni transazione è creata da un nodo e viene firmata digitalmente utilizzando una coppia di chiavi crittografiche (crittografia asimmetrica). \\
Prima di essere registrata nella blockchain, ogni transazione viene validata dalla rete secondo regole condivise e, una volta confermata, viene inserita in un blocco.

\subsubsection{Blocchi}
Un blocco è una struttura dati che aggrega un insieme di transazioni validate in un determinato intervallo di tempo. Oltre alle transazioni, ogni blocco contiene un \textit{header} con informazioni di gestione fondamentali, tra cui:
\begin{itemize} 
	\item l’\textit{hash} del blocco precedente, che collega crittograficamente i blocchi tra loro formando la catena;  
	\item un \textit{timestamp}, che indica il momento di creazione del blocco; 
	\item il \textit{Merkle Root}, ovvero la radice di un \textit{Merkle Tree}, una struttura ad albero binario che consente di riassumere tutte le transazioni del blocco in un unico \textit{hash};
	\item dati aggiuntivi richiesti dal protocollo di consenso adottato.
	\end{itemize}

L’inclusione dell’\textit{hash} del blocco precedente rende la blockchain una struttura immutabile per costruzione: anche una minima modifica a una singola transazione altererebbe la \textit{Merkle Root} e di conseguenza l’\textit{hash} del blocco, invalidando l’intera catena successiva. Questo meccanismo costituisce uno dei principali fattori di sicurezza della blockchain \cite{monrat2019survey}.


\subsubsection{Protocolli di consenso}
Come precedentemente affermato, la blockchain è un sistema intrinsecamente decentralizzato, per cui la validazione delle transazioni non dipende più da un'autorità centrale ma è subordinata all'approvazione collettiva da parte tutti i nodi della rete.
Il processo di validazione di un blocco impone dunque l'implementazione di algoritmi di consenso, come:

\begin{itemize}
	\item \gls{pow}: rappresenta il protocollo di consenso più noto e storicamente rilevante (introdotto da Bitcoin). In questo meccanismo, i nodi della rete, detti \textit{miner}, competono per risolvere un problema crittografico computazionalmente complesso ma facilmente verificabile. Il \textit{miner} che per primo individua una soluzione valida al problema, acquisisce il diritto di creare un nuovo blocco e propagarlo alla rete.
	Tuttavia, l'aggiunta effettiva del blocco avviene solo se questo viene correttamente validato e accettato dalla maggioranza dei nodi. Se queste operazioni vanno a buon fine, il \textit{miner} riceve una ricompensa, generalmente erogata sotto forma di criptovaluta \cite{tschorsch2016bitcoin}.
	
	\item \gls{pos}: in questo protocollo, il processo di validazione è affidato ad un insieme di nodi scelti in funzione della quantità di criptovaluta vincolata come garanzia, in un processo noto come \textit{staking}. I nodi selezionati come \textit{validatori} sono responsabili della creazione dei blocchi e della verifica delle transazioni. Una volta raggiunto il consenso, il blocco viene aggiunto immutabilmente alla blockchain, aggiornando lo stato del \textit{ledger} distribuito.
	
\end{itemize}


\subsubsection{Vantaggi della blockchain}
L'adozione della tecnologia blockchain offre vantaggi significativi rispetto ai sistemi centralizzati tradizionali, come:

\begin{itemize} 
	\item \textbf{Decentralizzazione}: la replicazione del registro su più nodi consente al sistema di operare correttamente anche in presenza di guasti, eliminando in questo modo i \textit{single point of failure} \cite{rodrigues2010peer};
	\item \textbf{Immutabilità}: ogni transazione è irreversibile, per cui una volta registrata nella blockchain, non può più essere cancellata o modificata, in quanto farlo altererebbe l’intera catena dei blocchi \cite{monrat2019survey}; 
	\item \textbf{Trasparenza}: tutte le transazioni sono verificabili pubblicamente (nelle \textit{blockchain permissionless}) o dai membri autorizzati (nelle \textit{permissioned}) \cite{crosby2016blockchain}; 
	\item \textbf{Sicurezza}: l’uso combinato di crittografia asimmetrica (per l'autenticazione) e protocolli di consenso distribuito (per l'integrità del registro) rende il sistema resistente ad attacchi e frodi;
	\end{itemize}
	

%-------------------------------------------------------------------------------------------------
\subsection{Ethereum Blockchain}
Ethereum è una piattaforma decentralizzata e \textit{open-source} basata su tecnologia blockchain, proposta da Vitalik Buterin nel 2013 e resa operativa con il primo rilascio stabile nel 2015 \cite{buterin2014next}.  \\
A differenza delle blockchain di prima generazione, concepite prevalentemente come registri distribuiti per il trasferimento di valore (come Bitcoin), Ethereum nasce come una piattaforma \textit{general-purpose} progettata per l’esecuzione di programmi direttamente sulla blockchain. \\
Questa estensione del modello tradizionale ha permesso di superare alcune limitazioni delle piattaforme precedenti, come la mancanza di Turing-completezza \cite{buterin2013ethereum}, e ha reso possibile la realizzazione di applicazioni decentralizzate, le \gls{dapps}, la cui logica applicativa è implementata tramite \textit{Smart Contract}.


\subsubsection{Smart Contract}
Gli \textit{Smart Contract} sono programmi immutabili memorizzati sulla blockchain di Ethereum. Come illustrato nel \textit{whitepaper} che ha introdotto questa tecnologia, gli \textit{Smart Contract} possono essere concepiti come entità crittografiche capaci di detenere e trasferire valore, il cui comportamento è rigidamente definito dal codice e la cui esecuzione è consentita esclusivamente al verificarsi delle condizioni prestabilite \cite{buterin2013ethereum}. \\
Per supportare la loro esecuzione, Ethereum adotta un modello basato su \textit{account}, distinguendo tra:
\begin{itemize}
	\item \textit{\gls{eoa}}: controllati direttamente dagli utenti tramite una coppia di chiavi crittografiche. La chiave privata è utilizzata per firmare digitalmente le transazioni, mentre la chiave pubblica funge da indirizzo dell’account sulla rete Ethereum. Una transazione è valida solo se firmata dal mittente, che ne garantisce l’autenticità e l’integrità;
	\item \textit{Contract Accounts}: account associati a uno \textit{Smart Contract}, privi di chiave privata. Essi non possono avviare transazioni, ma vengono attivati esclusivamente in risposta a chiamate provenienti da un \gls{eoa} o da un altro \textit{Smart Contract}.

\end{itemize}
Ogni interazione con uno \textit{Smart Contract} avviene tramite una transazione e comporta l’esecuzione di istruzioni all’interno dell'\gls{evm}. Il costo computazionale associato a tale esecuzione è quantificato in \textit{gas} ed è sostenuto dal mittente della transazione sotto forma di Ether, criptovaluta nativa di Ethereum. 


\subsubsection{DApps}
Le \gls{dapps} sono applicazioni decentralizzate che operano su una rete \gls{p2p}. A differenza delle applicazioni tradizionali, la logica di \textit{backend} non risiede su un server centralizzato, ma è implementata mediante \textit{Smart Contract} \cite{renu2021implementation}. \\
L’esecuzione di una DApp, infatti, avviene all’interno di una \gls{vm} distribuita: quando una funzione di uno \textit{Smart Contract} viene invocata tramite una transazione, ogni nodo della rete esegue la \gls{vm} ed elabora il codice del contratto in modo autonomo e deterministico. \\
Questa architettura conferisce alle \gls{dapps} vantaggi significativi rispetto ai sistemi centralizzati in termini di affidabilità, trasparenza e immutabilità \cite{metcalfe2020ethereum}.


%-------------------------------------------------------------------------------------------------
\subsection{Blockchain e Crowdfunding}





