% !TeX spellcheck = it_IT
% !TEX TS-program = pdflatex
% !TEX root = ../main.tex


% ********************************************************************
\section{Background}
\label{sec:background}
% ********************************************************************

\subsection{Tecnologia blockchain}

Una blockchain è una base di dati distribuita, condivisa e immutabile, mantenuta da una rete di nodi.

Nata come infrastruttura di supporto per la criptovaluta Bitcoin \cite{nakamoto2007bitcoin}, la blockchain si è progressivamente evoluta in una tecnologia \textit{general-purpose}, trovando applicazione in un'ampia gamma di contesti.

Come suggerito dal termine stesso, la blockchain è una catena cronologicamente ordinata di blocchi, dove ciascun blocco contiene un insieme di transazioni validate dalla rete.
Ogni blocco è collegato al precedente attraverso un riferimento crittografico, detto \textit{hash}.
Questa struttura dati rende estremamente difficile la modifica o la cancellazione delle informazioni già registrate: qualsiasi tentativo di alterare un blocco comporterebbe infatti la modifica di tutti i blocchi successivi, rendendo l’attacco facilmente rilevabile e computazionalmente impraticabile.


\subsubsection{Transazioni}
La transazione rappresenta l’unità fondamentale di informazione all’interno della blockchain. Essa descrive un’operazione richiesta da un utente, come ad esempio il trasferimento di un asset digitale.
Ogni transazione è creata da un nodo della rete e viene firmata digitalmente utilizzando una coppia di chiavi crittografiche.
Prima di essere registrata nella blockchain, ogni transazione viene validata dalla rete secondo regole condivise e, una volta confermata, viene inserita in un blocco che è crittograficamente collegato ai blocchi precedenti. 

\subsubsection{Blocchi}
Un blocco è una struttura dati che aggrega un insieme di transazioni validate in un determinato intervallo di tempo.
Oltre alle transazioni, ogni blocco contiene informazioni di gestione fondamentali, tra cui: l

\begin{itemize}
	\item l’hash del blocco precedente, che collega crittograficamente i blocchi tra loro; 
	\item un timestamp, che indica il momento di creazione del blocco;
	\item dati aggiuntivi richiesti dal protocollo di consenso adottato.
\end{itemize}

L’inclusione dell’hash del blocco precedente rende la blockchain una struttura immutabile per costruzione: anche una minima modifica a una singola transazione altererebbe l’hash del blocco, invalidando l’intera catena successiva. Questo meccanismo costituisce uno dei principali fattori di sicurezza della blockchain.

\subsubsection{Consenso e costruzione della blockchain}
In una rete blockchain decentralizzata, più nodi possono proporre contemporaneamente nuovi blocchi. Per determinare quale blocco debba essere aggiunto alla catena in modo univoco e condiviso, viene utilizzato un protocollo di consenso distribuito. I
l consenso è un algoritmo che permette alla rete di concordare su un’unica versione valida del ledger, anche in presenza di nodi malfunzionanti o potenzialmente malevoli.

Il primo protocollo di consenso largamente diffuso è la \gls{pow}, introdotta da Bitcoin, in cui i nodi competono per risolvere un problema computazionale complesso. Il nodo che risolve per primo il problema ottiene il diritto di aggiungere il nuovo blocco alla blockchain e di propagarlo alla rete. 

Altri protocolli di consenso, come Proof of Stake o meccanismi basati su votazione, sono stati successivamente sviluppati per migliorare efficienza energetica, latenza e scalabilità.
Una volta che un blocco è stato validato e accettato dalla maggioranza dei nodi, esso viene aggiunto alla blockchain e diventa parte integrante del ledger.
Con il passare del tempo e l’aggiunta di nuovi blocchi, le informazioni contenute nei blocchi più vecchi diventano sempre più sicure, poiché la loro modifica richiederebbe il controllo di una grande porzione della potenza di calcolo o dei validatori della rete.


\subsubsection{Vantaggi della blockchain}

\begin{itemize}
	\item Decentralizzazione:
	\item Immutabilità: 
	\item Trasparenza e assenza di fiducia: 
	\item Sicurezza: 
\end{itemize}


% \subsection{Bitcoin}

\subsection{Ethereum Blockchain}

\subsubsection{Smart Contract}

\subsubsection{DApps}

\subsection{DAO}






%\subsubsection{Tipologie di blockchain}

%\begin{itemize}
%	\item pubblica (permissionless): 
%	\item privata (permissioned):
%	contenuto...
%\end{itemize}
%ProofOfWork


%Come suggerito dal termine stesso, la blockchain è una catena di blocchi tra loro concatenati. Ogni blocco è una struttura dati contenente un set di campi di gestione (l'hash del blocco precedente, il timestamp del blocco corrente e altri) e un insieme di transazioni. La transazione è l'elemento utilizzato  per rappresentare l'operazione di scambio di valori o di dati tra due partecipanti alla rete.



%A differenza dei sistemi tradizionali, che richiedono la presenza di un’autorità centrale o di un intermediario fidato per la validazione e la conservazione delle transazioni, la blockchain consente interazioni dirette tra le parti senza la necessità di fiducia reciproca. Questo risultato è ottenuto grazie all’uso combinato di crittografia asimmetrica, meccanismi di consenso distribuito e strutture dati a catena di blocchi, che rendono estremamente difficile la modifica retroattiva delle informazioni una volta che esse sono state registrate nel ledger.
%



