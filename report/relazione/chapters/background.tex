% !TeX spellcheck = en_US
% !TEX TS-program = pdflatex
% !TEX root = ../main.tex


% ********************************************************************
\section{Stato dell'arte}
\label{sec:background}
% ********************************************************************

\subsection{Tecnologia blockchain}

Una blockchain è una base di dati distribuita, condivisa e immutabile. Nata inizialmente come infrastruttura di supporto per la criptovaluta Bitcoin \cite{nakamoto2007bitcoin}, la blockchain si è evoluta in una tecnologia \textit{general-purpose}, trovando  applicazione in un'ampia gamma di contesti, oltre quello finanziario.

Alla base di questa tecnologia vi è il concetto di \textit{Distributed Ledger Technology} (DLT), un registro distribuito, condiviso e sincronizzato tra più nodi di una rete \textit{peer-to-peer }(P2P).
A differenza dei sistemi tradizionali, che richiedono la presenza di un’autorità centrale o di un intermediario fidato per la validazione e la conservazione delle transazioni, la blockchain consente interazioni dirette tra le parti senza la necessità di fiducia reciproca. Questo risultato è ottenuto grazie all’uso combinato di crittografia asimmetrica, meccanismi di consenso distribuito e strutture dati a catena di blocchi, che rendono estremamente difficile la modifica retroattiva delle informazioni una volta che esse sono state registrate nel ledger.

Ogni transazione viene validata dalla rete secondo regole condivise e, una volta confermata, viene inserita in un blocco che è crittograficamente collegato ai blocchi precedenti. Questo meccanismo garantisce proprietà fondamentali quali immutabilità, tracciabilità e resistenza alle manomissioni, rendendo la blockchain particolarmente adatta alla gestione di asset digitali, processi verificabili e interazioni distribuite tra attori indipendenti.



\subsubsection{Algoritmo di consenso}


\subsubsection{Crittografia e sicurezza}

\subsubsection{Vantaggi della blockchain}


\subsection{Ethereum Blockchain}

\subsubsection{Introduzione a Ethereum}

\subsubsection{Smart Contract}


\subsection{DApps}

\subsection{DAO}


