% !TeX spellcheck = it_IT
% !TEX TS-program = pdflatex
% !TEX root = ../main.tex


% ********************************************************************
\section{Background}
\label{sec:background}
% ********************************************************************

\subsection{Blockchain}

Una blockchain è una base di dati distribuita, condivisa e immutabile, mantenuta da una rete di nodi interconnessi secondo un'architettura \gls{p2p}. Dal punto di vista dei sistemi distribuiti, essa può essere descritta come un sistema che garantisce affidabilità e sicurezza replicando i dati e le operazioni su molteplici nodi indipendenti \cite{rodrigues2010peer}.
Tale paradigma architetturale è stato introdotto per la prima volta nel 2008, come infrastruttura di supporto per la criptovaluta bitcoin proposta da Satoshi Nakamoto, pseudonimo usato per identificare l'autore (o gli autori) del white paper di Bitcoin \cite{nakamoto2007bitcoin}.

Come suggerito dal termine stesso, la blockchain è una catena ordinata di blocchi, dove ciascun blocco contiene un insieme di transazioni validate dalla rete.
Ogni blocco è collegato al precedente attraverso un riferimento crittografico, detto \textit{hash}.
Questa struttura dati rende estremamente difficile la modifica o la cancellazione delle informazioni già registrate: qualsiasi tentativo di alterare un blocco comporterebbe infatti la modifica di tutti i blocchi successivi \cite{monrat2019survey}.

\subsubsection{Transazioni}
La transazione rappresenta l’unità fondamentale di informazione all’interno della blockchain. Essa descrive un’operazione richiesta da un utente, come ad esempio il trasferimento di un asset digitale (\textit{token}) o l'esecuzione di uno \textit{Smart Contract} \cite{crosby2016blockchain}.
Ogni transazione è creata da un nodo della rete e viene firmata digitalmente utilizzando una coppia di chiavi crittografiche (crittografia asimmetrica). \\
Prima di essere registrata nella blockchain, ogni transazione viene validata dalla rete secondo regole condivise e, una volta confermata, viene inserita in un blocco.

\subsubsection{Blocchi}

Un blocco è una struttura dati che aggrega un insieme di transazioni validate in un determinato intervallo di tempo. Oltre alle transazioni, ogni blocco contiene un \textit{header} con informazioni di gestione fondamentali, tra cui:
\begin{itemize} 
	\item l’\textit{hash} del blocco precedente, che collega crittograficamente i blocchi tra loro formando la catena;  
	\item un \textit{timestamp}, che indica il momento di creazione del blocco; 
	\item il \textit{Merkle Root}, ovvero la radice di un \textit{Merkle Tree}, una struttura ad albero binario che consente di riassumere tutte le transazioni del blocco in un unico \textit{hash};
	\item dati aggiuntivi richiesti dal protocollo di consenso adottato.
	\end{itemize}
L’inclusione dell’\textit{hash} del blocco precedente rende la blockchain una struttura immutabile per costruzione: anche una minima modifica a una singola transazione altererebbe la \textit{Merkle Root} e di conseguenza l’\textit{hash} del blocco, invalidando l’intera catena successiva. Questo meccanismo costituisce uno dei principali fattori di sicurezza della blockchain \cite{monrat2019survey}.

\subsubsection{Protocolli di consenso}

Come precedentemente affermato, la blockchain è un sistema intrinsecamente decentralizzato, per cui la validazione delle transazioni non dipende più da un'autorità centrale ma è subordinata all'approvazione collettiva da parte tutti i nodi della rete.
Il processo di validazione di un blocco impone dunque l'implementazione di algoritmi di consenso, come:

\begin{itemize}
	\item \gls{pow}: rappresenta il protocollo di consenso più noto e storicamente rilevante (introdotto da Bitcoin). In questo meccanismo, i nodi della rete, detti \textit{miner}, competono per risolvere un problema crittografico computazionalmente complesso ma facilmente verificabile. 
	Il \textit{miner} che per primo individua una soluzione valida al problema, acquisisce il diritto di creare un nuovo blocco e propagarlo alla rete.
	Tuttavia, l'aggiunta effettiva del blocco avviene solo se questo viene correttamente validato e accettato dalla maggioranza dei nodi. Se queste operazioni vanno a buon fine, il \textit{miner} riceve una ricompensa, generalmente erogata sotto forma di criptovaluta \cite{tschorsch2016bitcoin}.
	
	\item \gls{pos}: in questo protocollo, il processo di validazione è affidato ad un insieme di nodi scelti in funzione della quantità di criptovaluta vincolata come garanzia, in un processo noto come \textit{staking}. I nodi selezionati come \textit{validatori} sono responsabili della creazione dei blocchi e della verifica delle transazioni. 
	
	La sicurezza della rete è garantita da un sistema di disincentivi economici: qualora un validatore agisca in modo fraudolento o tenti di compromettere il sistema, incorre nella perdita parziale o totale dei fondi posti in \textit{staking}. 
	Una volta raggiunto il consenso, il blocco viene aggiunto immutabilmente alla blockchain, aggiornando lo stato del \textit{ledger} distribuito.
	
\end{itemize}




\subsubsection{Vantaggi della blockchain}
L'adozione della tecnologia blockchain offre vantaggi significativi rispetto ai sistemi centralizzati tradizionali, derivanti dalla sua architettura distribuita e dall'uso della crittografia:

\begin{itemize} 
	\item \textbf{Decentralizzazione}: l’assenza di un’autorità centrale elimina i \textit{single point of failure} \cite{rodrigues2010peer}: 
	\item \textbf{Immutabilità}: ogni transazione è irreversibile, per cui una volta registrata nella blockchain, non può più essere cancellata o modificata, in quanto farlo altererebbe l’intera catena dei blocchi \cite{monrat2019survey}; 
	\item \textbf{Trasparenza}: tutte le transazioni sono verificabili pubblicamente (nelle \textit{blockchain permissionless}) o dai membri autorizzati (nelle \textit{permissioned}) \cite{crosby2016blockchain}; 
	\item \textbf{Disintermediazione (\textit{Trustless})}: il consenso distribuito consente interazioni affidabili tra parti che non si conoscono, eliminando la necessità di intermediari \cite{nakamoto2007bitcoin};
	\item \textbf{Sicurezza}: l’uso combinato di crittografia asimmetrica (per l'autenticazione) e protocolli di consenso distribuito (per l'integrità del registro) rende il sistema resistente ad attacchi e frodi;
	\end{itemize}


\subsection{Ethereum Blockchain}

\subsubsection{Smart Contract}

\subsubsection{DApps}

\subsubsection{DAO}




%subsection{crowdfunding}
%Nata come infrastruttura di supporto per la criptovaluta Bitcoin, la blockchain si è progressivamente evoluta in una tecnologia \textit{general-purpose}, trovando applicazione in un'ampia gamma di contesti. \cite{panarello2018blockchain}.


