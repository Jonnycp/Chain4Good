% !TeX spellcheck = it_IT
% !TEX TS-program = pdflatex
% !TEX root = ../main.tex


% ********************************************************************
\section{Progettazione}
\label{sec:progettazione}
% ********************************************************************

\subsection{L'obiettivo di Chain4Good}

Chain4Good è una piattaforma decentralizzata di \gls{cf} nata per superare le criticità intrinseche dei sistemi di raccolta fondi tradizionali. 
Il suo obiettivo principale è restituire al donatore un ruolo attivo lungo l’intero ciclo di vita della donazione, mitigando il problema della limitata tracciabilità nell’utilizzo dei fondi tipico dei sistemi centralizzati. \\
A differenza dei modelli tradizionali, nei quali le risorse vengono trasferite integralmente all’Ente beneficiario al termine della raccolta, in Chain4Good l’erogazione dei fondi avviene in maniera incrementale ed è subordinata a un processo di approvazione decentralizzato. In tale contesto, lo sblocco delle risorse è vincolato all’espressione del consenso dei donatori sulle singole richieste di spesa.\\
E'\ importante sottolineare che tale meccanismo non è esente da potenziali comportamenti fraudolenti. La tecnologia \textit{blockchain}, difatti, non è in grado di garantire la veridicità dei dati forniti \textit{off-chain}, quali i preventivi allegati alle richieste di spesa. 
Tuttavia, essa consente di rendere l’intero processo di richiesta, approvazione ed erogazione delle risorse immutabile, trasparente e pubblicamente verificabile, grazie alla registrazione \textit{on-chain} di ogni operazione e di ogni trasferimento di fondi. In questo modo, al donatore è permesso di certificare la congruità tra gli obiettivi dichiarati e quelli effettivamente perseguiti.\\
Chain4Good, dunque, si propone come una piattaforma capace di ridefinire il concetto stesso di donazione, il quale non si configura più come un mero atto di fiducia, bensì come un processo intrinsecamente sicuro e verificabile in ogni sua fase.


%l'idea stessa di donazione evolve

% Ogni operazione di donazione, infatti, viene registrata in modo permanente sulla \textit{blockchain} e può essere verificata pubblicamente, assicurando l'integrità delle informazioni e aumentando il livello di trasparenza. \\
% In tal senso, Chain4Good si distingue dallo stato dell’arte proponendosi come piattaforma orientata non solo alla raccolta fondi per fini prettamente filantropici, ma anche alla partecipazione e fiducia verificabile.



\subsection{Analisi dei requisiti}
In questa sezione sono riportati i requisiti richiesti per il corretto funzionamento della piattaforma. \\
In particolare, nella Tabella~\ref{tab:requisiti_funzionali} sono riportati i requisiti funzionali, ossia le funzionalità che la piattaforma deve implementare. \\
Nella Tabella \ref{tab:requisiti_non_funzionali}, invece, sono descritti i requisiti non funzionali i ovvero tutte le caratteristiche che, pur non essendo funzionalità dirette, il sistema deve garantire. 

\begin{table}
	\centering
	\footnotesize
	\renewcommand{\arraystretch}{1.5}
	\begin{tabularx}{\textwidth}{l p{4cm} X}
		\hline
		\textbf{ID} & \textbf{Requisito} & \textbf{Descrizione} \\
		\hline
		\multicolumn{3}{l}{\textbf{Ente}} \\
		\hline
		RF1 & Creazione progetto & L'Ente deve poter avviare una nuova iniziativa di raccolta fondi definendone nome, \textit{budget target} e data di scadenza. \\ \hline
		RF2 & Richiesta di spesa & L'Ente deve poter richiedere il rilascio di una parte dei fondi avanzando una richiesta di spesa e allegando il relativo preventivo. \\ \hline
		RF3 & Prova di acquisto & L'Ente deve poter caricare la fattura che attesti l'impiego dei fondi precedentemente sbloccati. \\ \hline
		RF4 & Vincolo di sequenzialità & L'Ente non deve poter sottomettere una nuova richiesta di spesa se non ha preventivamente caricato la prova di acquisto relativa alla richiesta precedentemente approvata. \\ \hline
		
		\multicolumn{3}{l}{\textbf{Donatore}} \\
		\hline
		RF5 & Visualizzazione progetti & Il donatore deve poter visualizzare l'elenco delle iniziative di \gls{cf} attive e i relativi dettagli. \\ \hline
		RF6 & Donazione & Il donatore deve poter selezionare un progetto e scegliere arbitrariamente l'importo da donare. \\ \hline
		RF7 & Votazione richieste & Il donatore deve poter visualizzare il preventivo di spesa allegato dall'Ente ed votare se approvare o meno la richiesta. \\ \hline
		RF8 & Saldo portafoglio & Il donatore deve poter visualizzare il saldo disponibile. \\ \hline
		
		\multicolumn{3}{l}{\textbf{Sistema}} \\
		\hline
		RF9 & Registrazione donazioni & Il sistema deve registrare \textit{on-chain} ogni donazione effettuata. \\ \hline
		RF10 & Blocco dei fondi & Il sistema deve impedire il trasferimento dei fondi, previo consenso dei donatori. \\ \hline
		RF11 & Gestione votazione & Il sistema deve avviare, gestire e concludere il processo di votazione per ogni richiesta di spesa. \\ \hline
		RF12 & Erogazione automatica & Il sistema deve trasferire automaticamente i fondi al \textit{wallet} dell'Ente, qualora la richiesta di spesa venga approvata dai donatori. \\ \hline
		RF13 & Registro operazioni & Il sistema deve registrare \textit{on-chain} le richieste di spesa, gli esiti delle votazioni e il trasferimento dei fondi sbloccati. \\ \hline
	\end{tabularx}
	\caption{Tabella dei requisiti funzionali.}
	\label{tab:requisiti_funzionali}
\end{table}



\begin{table}
	\centering
	\footnotesize
	\renewcommand{\arraystretch}{1.5}
	\begin{tabularx}{\textwidth}{l p{3cm} X}
		\hline
		\textbf{ID} & \textbf{Requisito} & \textbf{Descrizione} \\
		\hline
		RNF1 & Immutabilità & Ogni transazione relativa a donazioni, votazioni e rilascio di fondi deve essere registrata su un registro distribuito in modo permanente e non modificabile. \\ \hline
		RNF2 & Integrità dei dati & I file pesanti, come preventivi e fatture, devono essere memorizzati \textit{off-chain}. Il sistema deve garantire che tali documenti siano riconducibili in modo univoco alle relative operazioni registrate \textit{on-chain}, impedendone la manipolazione. \\ \hline
		RNF3 & Usabilità & La \textit{Webapp} deve consentire agli utenti di consultare i dati \textit{on-chain}, come lo storico delle donazioni effettuate, attraverso interfacce intuitive. \\ \hline
		RNF4 & Sicurezza & L'accesso alle funzionalità della piattaforma e alla consultazione dettagliata dei dati deve essere limitato ai soli utenti autenticati. \\ \hline
		RNF5 & Portabilità & La \textit{Webapp} deve essere fruibile sia da dispositivi \textit{desktop} che \textit{mobile}. \\ \hline
	\end{tabularx}
	\caption{Tabella dei requisiti non funzionali.}
	\label{tab:requisiti_non_funzionali}
\end{table}


% sicurezza: i dati devono rimanere all'interno dll'applicazione, protetti, deve essere sicuro replayact

\subsection{Architettura del Sistema}
L'architettura proposta adotta un modello che integra lo \textit{stack} MERN (MongoDB, Express, React, Node.js) con le tecnologie Web3.\\
In particolare, il \textit{front-end} è sviluppato in React integrato da Tailwind per la definizione di uno stile grafico resposivo, e da React Router per la gestione della navigazione \textit{client-side.}. \\
React Query, invece, è stato utilizzato per l'implementazione di un meccanismo di \textit{caching} voto ad ottimizzare le prestazioni e l'esperienza utente. \\
L'interazione con la blockchain è invece realizzata mediante le librerie Wagmi e Viem, attraverso le quali è stato possibile gestire l'interconnessione con il \textit{wallet} dell'utente (Metamask) e interagire direttamente con gli \textit{smart contract} per permettere l'esecuzione di transazioni e la lettura dei dati \textit{on-chain}.
 
Il \textit{back-end} è basato su Express.js ed è sviluppato in TypeScript. Inoltre, esso è strutturato secondo il \textit{pattern} \gls{mvc}, adattato a un’architettura API-based. \\
In particolare, il livello di \textit{Model} comprende i modelli Mongoose e l’accesso al database MongoDB, utilizzato per la persistenza dei dati \textit{off-chain}.
Il livello di \textit{Controller} è costituito dalle funzioni associate ai singoli \textit{endpoint} REST, responsabili dell’implementazione della logica applicativa. \\
Infine, il livello di \textit{View} è rappresentato dal sistema di \textit{routing} delle API, che espone le risorse verso il \textit{front-end}. \\
Tra il livello di \textit{routing} delle API e i controller sono inseriti specifici \textit{middleware}, incaricati di effettuare controlli di autenticazione e autorizzazione prima dell’esecuzione della logica applicativa (ad esempio, il \textit{middleware} \texttt{isEnte} controlla che l’utente autenticato possieda i requisiti necessari per operare come ente promotore).

Infine, per evidenti ragioni di efficienza e costi imposti dalla tecnologia blockchain è stata operata una chiara separazione tra i dati da memorizzare \textit{on-chain} e \textit{off-chain}. 
Nello specifico, nella \textbf{blockchain} è stato scelto di memorizzare i dati essenziali e critici, come l'insieme delle donazioni effettuate o le richieste di spesa avanzate dall'ente. Il database MongoDB, invece, è stato utilizzato per memorizzare rispettivamente: i metadati dei progetti e delle spese (come la descrizione dettagliata delle richieste di spesa), la logica non critica, i dati dell'utente e i documenti (.pdf) relativi alle prove di acquisto. 


Tale approccio ha consentito di combinare le garanzie di integrità e immutabilità offerte dalla blockchain con la flessibilità e l’efficienza delle infrastrutture web tradizionali.



\subsection{Stack tecnologico}
In questa sezione viene fornita una panoramica delle tecnologie impiegate per lo sviluppo della piattaforma. \\
In particolare, le tabelle \ref{tab:frontend_stack} e \ref{tab:backend_stack} riportano gli strumenti utilizzati rispettivamente per l’implementazione della componente \textit{front-end} e \textit{back-end} del \textit{software}, con una breve descrizione del loro impiego durante le fasi di sviluppo. Le tecnologie relative alla componente \textit{blockchain} sono invece sintetizzate in Tabella~\ref{tab:blockchain_stack}.\\
Infine, nella Tabella~\ref{tab:development_tools} sono riportati gli strumenti utilizzati per supportare l'organizzazione del lavoro tra i membri del gruppo e per garantire la portabilità del sistema \textit{software}.\\


%--------------------FRONTEND---------------------
\begin{table}
	\centering
	\small
	\renewcommand{\arraystretch}{1.3}
	\setlength{\tabcolsep}{12pt}
	\resizebox{\textwidth}{!}{
		\begin{tabular}{p{4cm} p{9cm}}
			\hline
			\textbf{Tecnologia} & \textbf{Descrizione} \\
			\hline
			TypeScript & Linguaggio utilizzato per lo sviluppo del \textit{front-end}. \\
			\hline
			React.js & Libreria utilizzata per la realizzazione dell’interfaccia utente secondo un’architettura a componenti. \\
			\hline
			React Router (7.11.0) & Libreria di \textit{routing} configurata in modalità \textit{framework} per la gestione della navigazione \textit{client-side}. \\
			\hline
			@tanstack/react-query & Libreria utilizzata per la gestione delle chiamate asincrone al \textit{back-end}, con supporto a \textit{caching}, sincronizzazione dei dati e gestione automatica degli stati di caricamento ed errore. \\
			\hline
			Tailwind CSS & \textit{Framework} CSS utilizzato per la definizione dello stile grafico e la realizzazione di \textit{layout} responsivi e coerenti, integrando una configurazione personalizzata. \\ 
			\hline
			Vite & Strumento di \textit{build} utilizzato per la compilazione del codice TypeScript e l’esecuzione dell’applicazione durante la fase di sviluppo. \\
			% il bundler è un componente specifico del processo di build-la sua funzione è analizzare l'albero delle dipendenze del progetto, quindi tutti gli import ed export, e unire centinaia di piccoli file in pochi file JavaScript
			\hline
			Wagmi & Libreria utilizzata per l’interazione con i \textit{wallet} e la \textit{blockchain} (Web3) attraverso \textit{hooks} in React. \\
			\hline
			viem & \textit{Client} RPC a basso livello, utilizzato internamente da Wagmi per il recupero dei dati \textit{on-chain}. \\
			\hline
			SIWE & Libreria utilizzata per l’implementazione del protocollo \textit{Sign-In with Ethereum}, adottato per verificare l’identità dell’utente tramite firma crittografica del \textit{wallet}. \\
			% protocollo EIP-4361 di autenticazione adottato per verificare l’identità dell’utente tramite firma crittografica del \textit{wallet}. \\
			%verifica se protocollo o libreria
			\hline
		\end{tabular}
	}
	\caption{Tecnologie e librerie utilizzate per lo sviluppo del \textit{front-end}.}
	\label{tab:frontend_stack}
\end{table}
\clearpage



%--------------------BACKEND---------------------
\begin{table}
	\centering
	\small
	\renewcommand{\arraystretch}{1.3}
	\setlength{\tabcolsep}{12pt}
	\resizebox{\textwidth}{!}{
		\begin{tabular}{lp{9cm}}
			\hline
			\textbf{Tecnologia} & \textbf{Descrizione} \\
			\hline
			TypeScript & Linguaggio utilizzato per lo sviluppo del \textit{back-end}, al fine di garantire tipizzazione statica e maggiore robustezza del codice. \\
			\hline
			Express.js & \textit{Framework} per Node.js utilizzato per la realizzazione delle API REST e per la gestione delle richieste provenienti dal \textit{front-end}. \\
			\hline
			MongoDB & Database NoSQL utilizzato per la memorizzazione dei dati \textit{off-chain}, quali metadati dei progetti, informazioni sugli utenti e dati di sessione. \\
			\hline
			Mongoose & \textit{\gls{odm}} utilizzato per la definizione dei modelli di dati e l’interazione con il database MongoDB. \\
			\hline
			express-session & \textit{Middleware} utilizzato per la gestione delle sessioni lato \textit{server}, impiegato nel processo di autenticazione. \\
			\hline
			SIWE & Libreria utilizzata per l’implementazione del protocollo \textit{Sign-In with Ethereum}, basato sulla verifica di messaggi firmati tramite \textit{wallet}. \\
			\hline
			ethers.js & Libreria JavaScript utilizzata per l’interazione con la \textit{blockchain} Ethereum, in particolare per il recupero di informazioni \textit{on-chain}. \\
			\hline
			Multer & \textit{Middelware} per Node.js  utilizzato per la gestione di flussi \textit{multipart/form-data}, facilitando l'\textit{upload} e il salvataggio dei contenuti multimediali. \\
			\hline
		\end{tabular}
	}
	\caption{Tecnologie e librerie utilizzate per lo sviluppo del \textit{back-end}.}
	\label{tab:backend_stack}
\end{table}
\clearpage


%--------------------BLOCKCHAIN---------------------
\begin{table}
	\centering
	\small
	\renewcommand{\arraystretch}{1.3}
	\setlength{\tabcolsep}{12pt}
	\resizebox{\textwidth}{!}{
		\begin{tabular}{lp{9cm}}
			\hline
			\textbf{Tecnologia} & \textbf{Descrizione} \\
			\hline
			Solidity & Linguaggio utilizzato per lo sviluppo degli \textit{Smart Contract}. \\
			\hline
			Hardhat & Ambiente di sviluppo utilizzato per la compilazione, il \textit{testing} e il \textit{deployment} degli \textit{Smart Contract}, nonché per la simulazione di una \textit{blockchain} locale. \\
			\hline
			@openzeppelin/contracts & Libreria di \textit{Smart Contract} riutilizzabili, utilizzata per integrare componenti standard e meccanismi di sicurezza.\\
			\hline
			Mocha & \textit{Framework} di \textit{testing} utilizzato come motore per l'esecuzione dei \textit{test} automatizzati degli \textit{Smart Contract}. \\
			\hline
			Chai & Libreria di asserzione utilizzata per verificare che l'\textit{output} degli \textit{Smart Contract} corrisponda ai requisiti attesi.\\
			\hline
		\end{tabular}
	}
	\caption{Tecnologie utilizzate per la componente \textit{blockchain}.}
	\label{tab:blockchain_stack}
\end{table}


% Mocha e Chai (libreria asse) utilizzate come librerie di test utilizzate per testare le funzioni del contratto prima di fare il deploy 
% Mocha è un test runner.
% Si occupa di eseguire i test, gestire le suite (describe, it), il setup/teardown, e mostrare i risultati. Non si occupa delle asserzioni (cioè dei controlli vero/falso).
%Chai - è  una libreria di asserzioni; fornisce funzioni come expect, assert, should per verificare che i risultati siano quelli attesi. Si integra facilmente con Mocha.



%--------------------SVILUPPO---------------------
\begin{table}
	\centering
	\small
	\renewcommand{\arraystretch}{1.3}
	\setlength{\tabcolsep}{12pt}
	\resizebox{\textwidth}{!}{
		\begin{tabular}{lp{9cm}}
			\hline
			\textbf{Strumento} & \textbf{Descrizione} \\
			\hline
			Visual Studio Code (1.107.1) & \gls{ide} principale, configurato con estensioni specifiche per lo sviluppo dello \textit{stack} tecnologico. \\
			\hline
			Git (2.52.0) & \textit{\gls{vcs}} impiegato per la gestione del codice sorgente secondo una strategia di \textit{branching} collaborativa per permettere lo sviluppo parallelo. \\
			\hline
			GitHub & Piattaforma di \textit{hosting} del \textit{repository} remoto, utilizzata per supportare la collaborazione tra i membri del gruppo. \\
			\hline
			Docker (29.1.3) & Piattaforma di containerizzazione utilizzata per la standardizzazione e l’isolamento dell’ambiente di esecuzione. \\
			\hline
		\end{tabular}
	}
	\caption{Strumenti utilizzati per lo sviluppo.}
	\label{tab:development_tools}
\end{table}


%docker compose: 1 container per frontend 1 per backend 1 per blockchain 









