% !TeX spellcheck = it_IT
% !TEX TS-program = pdflatex
% !TEX root = ../main.tex


% ********************************************************************
\section{Progettazione e implementazione}
\label{sec:progettazione}
% ********************************************************************

\subsection{L'obiettivo di Chain4Good}

%Nel settore del crowdfunding benefico, la fiducia tra enti promotori e donatori rappresenta un fattore critico, soprattutto per la gestione e la destinazione effettiva dei fondi raccolti. Le piattaforme di \textit{crowdfunding} tradizionali operano prevalentemente come intermediari centralizzati e, nella maggior parte dei casi, non forniscono strumenti sufficienti per garantire una tracciabilità completa e verificabile delle spese sostenute. Di conseguenza, il donatore dispone di informazioni limitate sull’impatto concreto del proprio contributo e deve affidarsi alla buona fede dell’ente promotore o della piattaforma stessa.

%\textit{Chain4Good} si colloca in questo contesto come una piattaforma di \textit{crowdfunding} decentralizzato che sfrutta la tecnologia blockchain per introdurre un modello di gestione dei fondi trasparente e verificabile. L’approccio adottato consente agli enti di suddividere le iniziative di raccolta in progetti distinti, ciascuno identificato univocamente sulla blockchain. Le donazioni effettuate dagli utenti vengono crittograficamente vincolate al progetto di riferimento attraverso uno \textit{Smart Contract}, eliminando la possibilità di utilizzi non autorizzati o non tracciabili delle risorse raccolte.

%A differenza dei modelli centralizzati, in Chain4Good i fondi non vengono immediatamente trasferiti all’ente promotore, ma sono custoditi all’interno di uno \textit{Smart Contract} che funge da \textit{vault} digitale. L’erogazione delle risorse è subordinata a un processo di approvazione decentralizzato, nel quale i donatori possono esprimere il proprio consenso sulle richieste di spesa presentate dall’ente. Questo meccanismo estende le proprietà di trasparenza, immutabilità e verificabilità della blockchain al livello applicativo del \textit{crowdfunding}, rafforzando il rapporto di fiducia tra le parti coinvolte.


%Chain4Good, che propone un’evoluzione del crowdfunding blockchain-based introducendo un controllo distribuito e continuo sull’utilizzo dei fondi, mediante una vault on-chain che blocca le donazioni e ne consente lo sblocco solo previa approvazione maggioritaria dei donatori, non pesata. L’integrazione di richieste di spesa documentate, votazioni motivate e verifica delle prove di acquisto realizza un modello che non si limita a garantire la trasparenza delle transazioni, ma estende la tracciabilità all’intero ciclo decisionale e operativo del progetto. In tal modo, Chain4Good si distingue dallo stato dell’arte come piattaforma orientata non solo alla raccolta, ma anche alla responsabilità, partecipazione e fiducia verificabile,


\subsection{Analisi dei requisiti}

\subsection{Analisi SWOT}

\subsection{Architettura del Software}
Prima di poter procedere alla progettazione dell’architettura del sistema da realizzare si è resa necessaria l’individuazione delle tecnologie da utilizzare in fase di sviluppo per poter comprendere come queste potessero interagire tra loro e soddisfare tutti i requisiti funzionali e non funzionali emersi dalla precedente fase di analisi.


