% !TeX spellcheck = it_IT
% !TEX TS-program = pdflatex
% !TEX root = ../main.tex


% ********************************************************************
\section{Progettazione e implementazione}
\label{sec:progettazione}
% ********************************************************************

\subsection{L'obiettivo di Chain4Good}

Chain4Good è una piattaforma decentralizzata di \gls{cf} nata per superare le criticità intrinseche dei sistemi di raccolta fondi tradizionali. 
Il suo obiettivo principale è restituire al donatore un ruolo attivo lungo l’intero ciclo di vita della donazione, mitigando il problema della limitata tracciabilità nell’utilizzo dei fondi tipico dei sistemi centralizzati. \\
A differenza dei modelli tradizionali, nei quali le risorse vengono trasferite integralmente all’Ente beneficiario al termine della raccolta, in Chain4Good l’erogazione dei fondi avviene in maniera incrementale ed è subordinata a un processo di approvazione decentralizzato. In tale contesto, lo sblocco delle risorse è vincolato all’espressione del consenso dei donatori sulle singole richieste di spesa.\\
E'\ importante sottolineare che tale meccanismo non è esente da potenziali comportamenti fraudolenti. La tecnologia \textit{blockchain}, difatti, non è in grado di garantire la veridicità dei dati forniti \textit{off-chain}, quali i preventivi allegati alle richieste di spesa. 
Tuttavia, essa consente di rendere l’intero processo di richiesta, approvazione ed erogazione delle risorse immutabile, trasparente e pubblicamente verificabile, grazie alla registrazione \textit{on-chain} di ogni operazione e di ogni trasferimento di fondi. In questo modo, al donatore è permesso di certificare la congruità tra gli obiettivi dichiarati e quelli effettivamente perseguiti.\\
Chain4Good, dunque, si propone come una piattaforma capace di ridefinire il concetto stesso di donazione, il quale non si configura più come un mero atto di fiducia, bensì come un processo intrinsecamente sicuro e verificabile in ogni sua fase.


%l'idea stessa di donazione evolve

% Ogni operazione di donazione, infatti, viene registrata in modo permanente sulla \textit{blockchain} e può essere verificata pubblicamente, assicurando l'integrità delle informazioni e aumentando il livello di trasparenza. \\
% In tal senso, Chain4Good si distingue dallo stato dell’arte proponendosi come piattaforma orientata non solo alla raccolta fondi per fini prettamente filantropici, ma anche alla partecipazione e fiducia verificabile.

\subsection{Analisi dei requisiti}
In questa sezione sono riportati per punti i requisiti richiesti per il corretto funzionamento della piattaforma.

\subsubsection{Requisiti funzionali}
I requisiti funzionali definiscono le funzionalità che la piattaforma deve implementare. Essi vengono di seguito categorizzati in base agli attori che interagiscono con il sistema. \\

\begin{enumerate}
	\item l'Ente Beneficiario:
	\begin{itemize}
		\item \textbf{Creazione progetto}: l'Ente deve poter avviare una nuova iniziativa di raccolta fondi definendone nome, \textit{budget target} e data di scadenza.
		\item \textbf{Inserimento di una richiesta di spesa}: l'Ente deve poter richiedere il rilascio di una parte dei fondi raccolti avanzando una richiesta di spesa e allegando il relativo preventivo;
		\item \textbf{Caricamento della prova di acquisto}: l'Ente deve poter caricare la fattura che attesti l'effettivo impiego dell'importo richiesto per lo scopo dichiarato;
		\item \textbf{Vincolo di sequenzialità sulle richieste di spesa}: l'Ente non deve poter sottomettere una nuova richiesta di spesa se non ha preventivamente caricato la prova di acquisto relativa alla richiesta precedentemente approvata;
	\end{itemize}
	
	
	\item per i Donatori:
	\begin{itemize}
		\item \textbf{Visualizzazione dei progetti:} il donatore deve poter visualizzare l'elenco delle iniziative di \gls{cf} attive e i relativi dettagli;
		\item \textbf{Donazione ad un progetto}: il donatore deve poter selezionare un progetto e scegliere arbitrariamente l'importo da donare;
		\item \textbf{Votazione delle richieste di spesa}: il donatore deve poter visualizzare il preventivo di spesa allegato dall'Ente ed esprimere una preferenza (se favorevole o contrario);
		\item \textbf{Visualizzazione del portafoglio}: il donatore deve poter visualizzare il saldo disponibile;
	\end{itemize}
	
	
	\item per il Sistema (logica implementata tramite \textit{Smart Contract}):
	\begin{itemize}
		\item \textbf{Registrazione delle donazioni}: il sistema deve registrare \textit{on-chain} ogni donazione effettuata;
		\item \textbf{Blocco dei fondi}: il sistema deve impedire il trasferimento dei fondi, previo consenso dei donatori;
		\item \textbf{Gestione del processo di votazione}: il sistema deve avviare, gestire e concludere il processo di votazione per ogni richiesta di spesa;
		\item \textbf{Erogazione automatica dei fondi}: il sistema deve trasferire automaticamente i fondi al \textit{wallet} dell'Ente, qualora la richiesta di spesa venga approvata dai donatori;
		\item \textbf{Registrazione delle operazioni}: il sistema deve registrare \textit{on-chain} le richieste di spesa, gli esiti delle votazioni e il trasferimento dei fondi sbloccati;
		
	\end{itemize}
	
\end{enumerate}

%il sistema deve consentire esclusivamente agli Enti autorizzati di avviare nuove iniziative di raccolta fondi.

% Descrizione dei requisiti funzionali mediante diagramma UML dei casi d'uso


\subsubsection{Requisiti non funzionali}
Di seguito si riporta l'elenco dei requisiti non funzionali, ossia tutte le caratteristiche che pur non essendo funzionalità, il sistema deve garantire. 

\begin{enumerate}
	
	\item \textbf{Immutabilità}: ogni transazione relativa a donazioni, votazioni e rilascio di fondi deve essere registrata su un registro distribuito in modo permanente e non modificabile;
	
	\item \textbf{Integrità dei dati}: i file pesanti, come preventivi e prove d'acquisto, devono essere memorizzati \textit{off-chain}. Il sistema deve garantire che tali documenti siano riconducibili in modo univoco alle relative operazioni registrate \textit{on-chain}, impedendone la manipolazione;
	
	\item \textbf{Usabilità}: la \textit{Webapp} deve consentire agli utenti di consultare i dati \textit{on-chain}, come lo storico delle donazioni effettuate, attraverso interfacce intuitive;
	
	\item \textbf{Sicurezza}: l'accesso alle funzionalità della piattaforma e alla consultazione dettagliata dei dati deve essere limitato ai soli utenti autenticati;
	
	\item \textbf{Portabilità}: la \textit{Webapp} deve essere fruibile sia da dispositivi \textit{desktop} che \textit{mobile};
\end{enumerate}




%il sistema deve garantire la sicurezza: deve essre accessibile sia da computer che da telefono
% sicurezza: i dati devono rimanere all'interno dll'applicazione, protetti, deve essere sicuro replayact
%forse sicurezza: i dati stanno nell'applicazione e non sono accessibili a chi non è loggato; per vedere i dati devi essere autenticato 
%




     % \item \textbf{Architettura decentralizzata}: il sistema non deve presentare \gls{spof}, per cui anche in caso di down della \textit{Webapp}, i fondi devono rimanere accessibili e gestibili;




\subsection{Architettura del Sistema}
Prima di poter procedere alla progettazione dell’architettura del sistema da realizzare si è resa necessaria l’individuazione delle tecnologie da utilizzare in fase di sviluppo per poter comprendere come queste potessero interagire tra loro e soddisfare tutti i requisiti funzionali e non funzionali emersi dalla precedente fase di analisi.



\subsection{Stack tecnologico}

\subsubsection{Front-end}
La Tabella~\ref{tab:frontend_stack} riassume le principali tecnologie adottate per lo sviluppo del front-end della piattaforma.

\smallskip
\begin{table}[h]
	\centering
	\small
	\renewcommand{\arraystretch}{1.3}
	\setlength{\tabcolsep}{12pt}
	\resizebox{\textwidth}{!}{
		\begin{tabular}{p{4cm} p{9cm}}
			\hline
			\textbf{Tecnologia} & \textbf{Descrizione} \\
			\hline
			TypeScript & Linguaggio utilizzato per lo sviluppo del \textit{front-end}. \\
			\hline
			React JS & Libreria utilizzata per la realizzazione dell’interfaccia utente secondo un’architettura a componenti. \\
			\hline
			React Router & Libreria impiegata per la gestione della navigazione \textit{client-side}. \\
			\hline
			@tanstack/react-query & Libreria utilizzata per la gestione delle chiamate asincrone al \textit{backend}, con supporto a \textit{caching}, sincronizzazione dei dati e gestione automatica degli stati di caricamento ed errore. \\
			\hline
			Tailwind CSS & \textit{Framework} CSS utilizzato per la definizione dello stile grafico e la realizzazione di \textit{layout} responsivi e coerenti. \\
			\hline
			Vite & Strumento di \textit{build} e sviluppo utilizzato per la compilazione del codice TypeScript e l’esecuzione dell’applicazione durante la fase di sviluppo. \\
			\hline
			Wagmi & Libreria utilizzata per l’integrazione Web3 e l’interazione con \textit{wallet} Ethereum e \textit{Smart Contract} tramite \textit{hook React}. \\
			\hline
			viem & \textit{Client} RPC a basso livello, utilizzato internamente per la comunicazione con i nodi \textit{blockchain} e il recupero dei dati \textit{on-chain}. \\
			\hline
			SIWE & Protocollo di autenticazione adottato per verificare l’identità dell’utente tramite firma crittografica del \textit{wallet}. \\
			\hline
		\end{tabular}
	}
	\caption{Tecnologie e librerie utilizzate per lo sviluppo del front-end.}
	\label{tab:frontend_stack}
\end{table}
\clearpage

\subsubsection{Back-end}
La Tabella~\ref{tab:backend_stack} riassume le principali tecnologie e librerie adottate per lo sviluppo del \textit{back-end} della piattaforma.

\smallskip
\begin{table}[h]
	\centering
	\small
	\renewcommand{\arraystretch}{1.3}
	\setlength{\tabcolsep}{12pt}
	\resizebox{\textwidth}{!}{
		\begin{tabular}{p{4cm} p{9cm}}
			\hline
			\textbf{Tecnologia} & \textbf{Descrizione} \\
			\hline
			TypeScript & Linguaggio utilizzato per lo sviluppo del \textit{back-end}, al fine di garantire tipizzazione statica e maggiore robustezza del codice. \\
			\hline
			Express.js & \textit{Framework} per Node.js utilizzato per la realizzazione delle API REST e per la gestione delle richieste provenienti dal \textit{front-end.} \\
			\hline
			MongoDB & Database NoSQL utilizzato per la memorizzazione dei dati \textit{off-chain}, quali metadati dei progetti, informazioni sugli utenti e dati di sessione. \\
			\hline
			Mongoose & \textit{\gls{odm}} utilizzato per la definizione dei modelli di dati e l’interazione con il database MongoDB. \\
			\hline
			express-session & \textit{Middleware} utilizzato per la gestione delle sessioni lato \textit{server}, impiegato nel processo di autenticazione degli utenti. \\
			\hline
			SIWE & Libreria utilizzata per l’implementazione del protocollo \textit{Sign-In with Ethereum}, basato sulla verifica di messaggi firmati tramite \textit{wallet} Ethereum. \\
			\hline
			ethers.js & Libreria JavaScript utilizzata per l’interazione con la \textit{blockchain} Ethereum, in particolare per il recupero di informazioni \textit{on-chain}. \\
			\hline
		\end{tabular}
	}
	\caption{Tecnologie e librerie utilizzate per lo sviluppo del \textit{back-end}.}
	\label{tab:backend_stack}
\end{table}
\clearpage

\subsubsection{Blockchain}
La Tabella~\ref{tab:blockchain_stack} riassume le tecnologie e le librerie adottate per l'implementazione della componente \textit{blockchain} del sistema.

\smallskip
\begin{table}[h]
	\centering
	\small
	\renewcommand{\arraystretch}{1.3}
	\setlength{\tabcolsep}{12pt}
	\resizebox{\textwidth}{!}{
		\begin{tabular}{p{4cm} p{9cm}}
			\hline
			\textbf{Tecnologia} & \textbf{Descrizione} \\
			\hline
			Solidity & Linguaggio utilizzato per lo sviluppo degli \textit{Smart Contract}. \\
			\hline
			Hardhat & Ambiente di sviluppo utilizzato per la compilazione, il \textit{testing} e il \textit{deployment} degli \textit{Smart Contract}, nonché per la simulazione di una \textit{blockchain} locale. \\
			\hline
			@openzeppelin/contracts & Libreria di \textit{Smart Contract} riutilizzabili, utilizzata per integrare componenti standard e meccanismi di sicurezza, come il controllo degli accessi. \\
			\hline
		\end{tabular}
	}
	\caption{Tecnologie utilizzate per la componente \textit{blockchain}.}
	\label{tab:blockchain_stack}
\end{table}
\smallskip




% Oltre all’impiego di linguaggi, librerie e ambienti di sviluppo per la realizzazione degli smart contract, il progetto prevede la definizione di specifiche componenti applicative on-chain che implementano la logica funzionale del sistema di crowdfunding. Tali componenti non costituiscono tecnologie a sé stanti, ma rappresentano strutture e meccanismi progettati per rispondere ai requisiti applicativi del dominio.
%In particolare, la custodia dei fondi raccolti è affidata a un contratto dedicato, indicato come \textit{Smart Contract Vault}, che svolge la funzione di contenitore decentralizzato delle donazioni. Le risorse economiche versate dagli utenti non vengono trasferite immediatamente al promotore del progetto, ma rimangono bloccate all’interno del vault fino al verificarsi delle condizioni previste, quali l’approvazione di una richiesta di spesa. Questo approccio consente di separare la gestione dei fondi dalla logica off-chain e di vincolare l’utilizzo delle risorse alle regole definite nel codice dello smart contract.
% Il collegamento tra i dati gestiti off-chain e la logica on-chain è realizzato mediante l’introduzione di un identificativo univoco di progetto (\textit{ProjectID}). Tale identificativo funge da riferimento comune tra frontend, backend e smart contract, permettendo di associare in modo non ambiguo le informazioni descrittive di un progetto alle operazioni eseguite sulla blockchain. L’utilizzo del \textit{ProjectID} consente inoltre di garantire la verificabilità delle operazioni, permettendo agli utenti di correlare le informazioni visualizzate nell’interfaccia applicativa con le transazioni registrate on-chain.
% Infine, il progetto integra un meccanismo di voto on-chain per la gestione delle richieste di spesa. Tale meccanismo consente ai donatori di esprimere un’approvazione o un rifiuto in merito all’utilizzo dei fondi raccolti, secondo un processo di decisione collettiva implementato direttamente negli smart contract. Le votazioni e i relativi esiti sono registrati sulla blockchain, rendendo il processo trasparente, verificabile e resistente a manipolazioni. Lo sblocco dei fondi avviene esclusivamente al raggiungimento delle condizioni di consenso previste, garantendo un controllo condiviso sull’allocazione delle risorse.


%La componente blockchain del progetto è stata realizzata mediante lo sviluppo di \textit{Smart Contract} in linguaggio \textit{Solidity}, eseguiti su una \textit{blockchain} compatibile con la \gls{evm}.
% Per lo sviluppo, il \textit{testing} e il \textit{deployment} degli \textit{Smart Contract} è stato utilizzato l’ambiente \textit{Hardhat}, che consente la compilazione del codice \textit{Solidity}, l’esecuzione di test automatizzati e la simulazione di una rete \textit{blockchain} locale. 
% Al fine di garantire sicurezza e conformità agli standard, gli \textit{Smart Contract}  fanno uso della libreria \texttt{@openzeppelin/contracts}, che fornisce implementazioni affidabili di componenti riutilizzabili, quali meccanismi di controllo degli accessi e interfacce standard per la gestione dei \textit{token}.



\subsubsection{Strumenti di sviluppo}
Visual Studio Code è stato utilizzato come \gls{ide} principale. \\
Per favorire lo sviluppo parallelo e la portabilità del sistema \textit{software}, invece, sono stati utilizzati strumenti riportati in Tabella~\ref{tab:development_tools}.

\smallskip
\begin{table}[h]
	\centering
	\small
	\renewcommand{\arraystretch}{1.3}
	\setlength{\tabcolsep}{12pt}
	\resizebox{\textwidth}{!}{
		\begin{tabular}{p{4cm} p{9cm}}
			\hline
			\textbf{Strumento} & \textbf{Descrizione} \\
			\hline
			Git & \textit{\gls{vcs}} impiegato per la gestione del codice sorgente secondo una strategia di \textit{branching} collaborativa per permettere lo sviluppo parallelo. \\
			\hline
			GitHub & Piattaforma di \textit{hosting} del \textit{repository} remoto, utilizzata per supportare la collaborazione tra i membri del gruppo. \\
			\hline
			Docker & Utilizzato per la standardizzazione e l’isolamento dell’ambiente di esecuzione. \\
			\hline
		\end{tabular}
	}
	\caption{Strumenti utilizzati per lo sviluppo del progetto}
	\label{tab:development_tools}
\end{table}



% è stata adottata la containerizzazione tramite \textit{Docker}, che ha permesso di standardizzare l’ambiente di esecuzione dell’applicazione, riducendo le dipendenze dall’infrastruttura sottostante e facilitando le fasi di distribuzione e avvio del sistema.

