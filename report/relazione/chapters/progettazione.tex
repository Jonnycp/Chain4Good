% !TeX spellcheck = it_IT
% !TEX TS-program = pdflatex
% !TEX root = ../main.tex


% ********************************************************************
\section{Progettazione e implementazione}
\label{sec:progettazione}
% ********************************************************************

\subsection{L'obiettivo di Chain4Good}

Chain4Good è una piattaforma decentralizzata di \gls{cf} nata per superare le criticità intrinseche dei sistemi di raccolta fondi tradizionali. 
Il suo obiettivo principale è restituire al donatore un ruolo attivo lungo l’intero ciclo di vita della donazione, mitigando il problema della limitata tracciabilità nell’utilizzo dei fondi tipico dei sistemi centralizzati. \\
A differenza dei modelli tradizionali in cui le risorse vengono trasferite integralmente all'Ente al termine della raccolta, infatti, in Chain4Good l’erogazione delle stesse è incrementale ed è subordinata ad un processo di approvazione decentralizzato, attraverso il quale donatori sono chiamati a esprimere il proprio consenso in merito alle richieste di spesa.
E'\ importante sottolineare che tale meccanismo non è esente da potenziali comportamenti fraudolenti. La tecnologia \textit{blockchain}, difatti, non è in grado di garantire la veridicità dei dati forniti \textit{off-chain}, quali i preventivi allegati alle richieste di spesa. 
Tuttavia, essa consente di rendere l’intero processo di richiesta, approvazione ed erogazione delle risorse immutabile, trasparente e pubblicamente verificabile, grazie alla registrazione \textit{on-chain} di ogni operazione e di ogni trasferimento di fondi. In questo modo, il donatore può certificare la congruità tra gli obiettivi dichiarati e quelli effettivamente perseguiti.
Chain4Good, dunque, si propone come una piattaforma capace di ridefinire il concetto stesso di donazione, il quale non si configura più come un mero atto di fiducia, bensì come un processo intrinsecamente sicuro e verificabile in ogni sua fase.


%Alla luce di tali considerazioni, l’adozione della tecnologia \textit{blockchain} permette di ridefinire il concetto stesso di donazione, che non si configura più come un mero atto di fiducia, ma come un processo intrinsecamente sicuro e verificabile in ogni sua fase.
%Ecco che quindi, adottando questa tecnologia, l'idea stessa di donazione evolve, trasformandosi da un mero atto di fiducia ad un processo intrinsecamente sicuro e verificabile in ogni sua fase. 


% Ogni operazione di donazione, infatti, viene registrata in modo permanente sulla \textit{blockchain} e può essere verificata pubblicamente, assicurando l'integrità delle informazioni e aumentando il livello di trasparenza. \\
% In tal senso, Chain4Good si distingue dallo stato dell’arte proponendosi come piattaforma orientata non solo alla raccolta fondi per fini prettamente filantropici, ma anche alla partecipazione e fiducia verificabile.



%\begin{table}
%	\centering
%	\begin{tabular}{ccc}
%		\toprule
%		\textbf{Caratteristica} & \textbf{CF tradizionale} & \textbf{Chain4Good} \\
%		\midrule
%		Gestione dei fondi & Centralizzata & Decentralizzata \\
%		Tracciabilità spese & Assente & Vincolata \\
%		Ruolo del donatore & Passivo & Attivo \\
%		Trasparenza & Limitata & Elevata \\
%   	\bottomrule
%   \end{tabular}
%	\caption{Confronto tra le piattaforme di CF esistenti e Chain4Good}
%	\label{tab:obiettivi}
%\end{table}




%Questo meccanismo estende le proprietà di trasparenza, immutabilità e verificabilità della \textit{blockchain} al livello applicativo dl \gls{cf}, rafforzando il rapporto di fiducia tra le parti coinvolte.




\subsection{Analisi dei requisiti}



\subsubsection{Requisiti funzionali}



%gestione degli enti e dei progetti: il sistema deve consentire a enti autorizzati di creare e gestire progetti di crowdfunding, specificando informazioni quali obiettivo di raccolta, descrizione, categoria, materiale multimediale ed eventuali vincoli temporali. Ogni progetto deve essere identificato in modo univoco all’interno del sistema

%Visualizzazione e consultazione dei progetti
%La piattaforma deve permettere agli utenti di visualizzare i progetti attivi e di consultare i relativi dettagli, al fine di supportare una scelta consapevole in fase di donazione.

%Donazione in criptovaluta
%Il sistema deve consentire agli utenti donatori di effettuare donazioni in criptovaluta, associando ogni transazione al progetto selezionato. Le donazioni devono essere registrate in modo permanente sulla blockchain tramite smart contract.

%Gestione dei fondi tramite smart contract
%I fondi raccolti devono essere trasferiti e custoditi all’interno di uno smart contract di tipo Vault, che ne impedisce l’utilizzo diretto fino all’approvazione di una richiesta di spesa, garantendo così il vincolo crittografico dei fondi al progetto di riferimento.

%Richiesta e approvazione delle spese
%Il sistema deve permettere all’ente promotore di un progetto di inviare richieste di spesa corredate da documentazione giustificativa. I donatori devono poter esprimere un voto di approvazione o rifiuto, con meccanismo di maggioranza semplice e senza ponderazione del voto.

%Erogazione dei fondi e rendicontazione
%In caso di approvazione della richiesta, il sistema deve trasferire automaticamente i fondi al wallet dell’ente. Dopo l’esecuzione della spesa, l’ente deve fornire una prova di acquisto prima di poter avanzare nuove richieste.

%Consultazione dello storico delle transazioni
%Il sistema deve consentire agli utenti di interrogare la blockchain tramite l’identificativo del progetto, visualizzando lo storico completo delle donazioni, delle richieste di spesa e delle transazioni effettuate.



\subsubsection{Requisiti non funzionali}


%\subsection{Analisi SWOT}



\subsection{Architettura del Sistema}
Prima di poter procedere alla progettazione dell’architettura del sistema da realizzare si è resa necessaria l’individuazione delle tecnologie da utilizzare in fase di sviluppo per poter comprendere come queste potessero interagire tra loro e soddisfare tutti i requisiti funzionali e non funzionali emersi dalla precedente fase di analisi.

\subsubsection{Architettura del Software}

% Backend e frontend


\subsubsection{Strumenti di sviluppo e deployment}

% è stata adottata la containerizzazione tramite \textit{Docker}, che ha permesso di standardizzare l’ambiente di esecuzione dell’applicazione, riducendo le dipendenze dall’infrastruttura sottostante e facilitando le fasi di distribuzione e avvio del sistema.

