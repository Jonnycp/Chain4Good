% !TeX spellcheck = it_IT
% !TEX TS-program = pdflatex
% !TEX root = ../main.tex


% ********************************************************************
\section{Progettazione e implementazione}
\label{sec:progettazione}
% ********************************************************************

\subsection{L'obiettivo di Chain4Good}

Chain4Good è una piattaforma decentralizzata di \gls{cf} nata per superare le criticità intrinseche dei sistemi di raccolta fondi tradizionali. 
Il suo obiettivo principale è restituire al donatore un ruolo attivo lungo l’intero ciclo di vita della donazione, mitigando il problema della limitata tracciabilità nell’utilizzo dei fondi tipico dei sistemi centralizzati. \\
A differenza dei modelli tradizionali in cui le risorse vengono trasferite integralmente all'Ente al termine della raccolta, infatti, in Chain4Good l’erogazione delle stesse è incrementale ed è subordinata ad un processo di approvazione decentralizzato, attraverso il quale donatori sono chiamati a esprimere il proprio consenso in merito alle richieste di spesa.
E'\ importante sottolineare che tale meccanismo non è esente da potenziali comportamenti fraudolenti. La tecnologia \textit{blockchain}, difatti, non è in grado di garantire la veridicità dei dati forniti \textit{off-chain}, quali i preventivi allegati alle richieste di spesa. 
Tuttavia, essa consente di rendere l’intero processo di richiesta, approvazione ed erogazione delle risorse immutabile, trasparente e pubblicamente verificabile, grazie alla registrazione \textit{on-chain} di ogni operazione e di ogni trasferimento di fondi. In questo modo, il donatore può certificare la congruità tra gli obiettivi dichiarati e quelli effettivamente perseguiti.
Chain4Good, dunque, si propone come una piattaforma capace di ridefinire il concetto stesso di donazione, il quale non si configura più come un mero atto di fiducia, bensì come un processo intrinsecamente sicuro e verificabile in ogni sua fase.


%Alla luce di tali considerazioni, l’adozione della tecnologia \textit{blockchain} permette di ridefinire il concetto stesso di donazione, che non si configura più come un mero atto di fiducia, ma come un processo intrinsecamente sicuro e verificabile in ogni sua fase.
%Ecco che quindi, adottando questa tecnologia, l'idea stessa di donazione evolve, trasformandosi da un mero atto di fiducia ad un processo intrinsecamente sicuro e verificabile in ogni sua fase. 


% Ogni operazione di donazione, infatti, viene registrata in modo permanente sulla \textit{blockchain} e può essere verificata pubblicamente, assicurando l'integrità delle informazioni e aumentando il livello di trasparenza. \\
% In tal senso, Chain4Good si distingue dallo stato dell’arte proponendosi come piattaforma orientata non solo alla raccolta fondi per fini prettamente filantropici, ma anche alla partecipazione e fiducia verificabile.



%\begin{table}
%	\centering
%	\begin{tabular}{ccc}
%		\toprule
%		\textbf{Caratteristica} & \textbf{CF tradizionale} & \textbf{Chain4Good} \\
%		\midrule
%		Gestione dei fondi & Centralizzata & Decentralizzata \\
%		Tracciabilità spese & Assente & Vincolata \\
%		Ruolo del donatore & Passivo & Attivo \\
%		Trasparenza & Limitata & Elevata \\
%   	\bottomrule
%   \end{tabular}
%	\caption{Confronto tra le piattaforme di CF esistenti e Chain4Good}
%	\label{tab:obiettivi}
%\end{table}




%Questo meccanismo estende le proprietà di trasparenza, immutabilità e verificabilità della \textit{blockchain} al livello applicativo dl \gls{cf}, rafforzando il rapporto di fiducia tra le parti coinvolte.


\subsection{Analisi dei requisiti}
In questa sezione sono riportati per punti i requisiti richiesti per il funzionamento della piattaforma.

\subsubsection{Requisiti funzionali}
I requisiti funzionali elencati definiscono il comportamento atteso della piattaforma Chain4Good.

\begin{enumerate}
	\item \textbf{Creazione progetto:} il sistema deve consentire esclusivamente agli Enti autorizzati di avviare nuove iniziative di raccolta fondi.
	\item \textbf{Consultazione dello stato dei progetti:}
	\item \textbf{Inserimento di una richiesta di spesa:} 
	\item \textbf{Caricamento della prova di acquisto:}
	\item \textbf{Vincolo di sequenzialità sulle richieste di spesa:} l'Ente non può sottomettere una nuova richiesta di spesa se non ha preventivamente caricato la prova di acquisto relativa alla richiesta precedentemente approvata;
	
\end{enumerate}


I requisiti non funzionali definiscono le proprietà qualitative del sistema Chain4Good.

\begin{enumerate}
	\item \textbf{Creazione progetto:} il sistema deve consentire esclusivamente agli Enti autorizzati di avviare nuove iniziative di raccolta fondi;
	\item \textbf{Rimozione di un progetto:}
	\item \textbf{Visualizzazione dei progetti attivi:}
	\item \textbf{}
	\item \textbf{}
	\item \textbf{}
	\item \textbf{}
	\item \textbf{}
	
\end{enumerate}





\subsubsection{Requisiti non funzionali}
Di seguito si riporta l'elenco dei requisiti non funzionali, ovvero tutte le caratteristiche che pur non essendo funzionalità, che il sistema deve garantire.





\subsection{Architettura del Sistema}
Prima di poter procedere alla progettazione dell’architettura del sistema da realizzare si è resa necessaria l’individuazione delle tecnologie da utilizzare in fase di sviluppo per poter comprendere come queste potessero interagire tra loro e soddisfare tutti i requisiti funzionali e non funzionali emersi dalla precedente fase di analisi.

\subsubsection{Architettura del Software}

% Backend e frontend


\subsubsection{Strumenti di sviluppo e deployment}

% è stata adottata la containerizzazione tramite \textit{Docker}, che ha permesso di standardizzare l’ambiente di esecuzione dell’applicazione, riducendo le dipendenze dall’infrastruttura sottostante e facilitando le fasi di distribuzione e avvio del sistema.

