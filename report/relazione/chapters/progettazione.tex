% !TeX spellcheck = it_IT
% !TEX TS-program = pdflatex
% !TEX root = ../main.tex


% ********************************************************************
\section{Progettazione e implementazione}
\label{sec:progettazione}
% ********************************************************************

\subsection{L'obiettivo di Chain4Good}

Chain4Good è una piattaforma nata per superare i limiti strutturali delle piattaforme di \gls{cf} esistenti. L'obiettivo primario è restituire al donatore un ruolo attivo sull'intero ciclo di vita della donazione, risolvendo il problema della mancata tracciabilità nell'utilizzo dei fondi proprio ai sistemi centralizzati. \\
L’erogazione delle risorse, infatti, è subordinata a un processo di approvazione, attraverso il quale i donatori possono esprimere il proprio consenso sulle richieste di spesa presentate dall’Ente.\\
In tal modo, Chain4Good si distingue dallo stato dell’arte come piattaforma orientata non solo alla raccolta, ma anche alla responsabilità, partecipazione e fiducia verificabile.




%Questo meccanismo estende le proprietà di trasparenza, immutabilità e verificabilità della \textit{blockchain} al livello applicativo dl \gls{cf}, rafforzando il rapporto di fiducia tra le parti coinvolte.




\subsection{Analisi dei requisiti}



\subsubsection{Requisiti funzionali}



%gestione degli enti e dei progetti: il sistema deve consentire a enti autorizzati di creare e gestire progetti di crowdfunding, specificando informazioni quali obiettivo di raccolta, descrizione, categoria, materiale multimediale ed eventuali vincoli temporali. Ogni progetto deve essere identificato in modo univoco all’interno del sistema

%Visualizzazione e consultazione dei progetti
%La piattaforma deve permettere agli utenti di visualizzare i progetti attivi e di consultare i relativi dettagli, al fine di supportare una scelta consapevole in fase di donazione.

%Donazione in criptovaluta
%Il sistema deve consentire agli utenti donatori di effettuare donazioni in criptovaluta, associando ogni transazione al progetto selezionato. Le donazioni devono essere registrate in modo permanente sulla blockchain tramite smart contract.

%Gestione dei fondi tramite smart contract
%I fondi raccolti devono essere trasferiti e custoditi all’interno di uno smart contract di tipo Vault, che ne impedisce l’utilizzo diretto fino all’approvazione di una richiesta di spesa, garantendo così il vincolo crittografico dei fondi al progetto di riferimento.

%Richiesta e approvazione delle spese
%Il sistema deve permettere all’ente promotore di un progetto di inviare richieste di spesa corredate da documentazione giustificativa. I donatori devono poter esprimere un voto di approvazione o rifiuto, con meccanismo di maggioranza semplice e senza ponderazione del voto.

%Erogazione dei fondi e rendicontazione
%In caso di approvazione della richiesta, il sistema deve trasferire automaticamente i fondi al wallet dell’ente. Dopo l’esecuzione della spesa, l’ente deve fornire una prova di acquisto prima di poter avanzare nuove richieste.

%Consultazione dello storico delle transazioni
%Il sistema deve consentire agli utenti di interrogare la blockchain tramite l’identificativo del progetto, visualizzando lo storico completo delle donazioni, delle richieste di spesa e delle transazioni effettuate.



\subsubsection{Requisiti non funzionali}


%\subsection{Analisi SWOT}



\subsection{Architettura del Sistema}
Prima di poter procedere alla progettazione dell’architettura del sistema da realizzare si è resa necessaria l’individuazione delle tecnologie da utilizzare in fase di sviluppo per poter comprendere come queste potessero interagire tra loro e soddisfare tutti i requisiti funzionali e non funzionali emersi dalla precedente fase di analisi.

\subsubsection{Architettura del Software}

% Backend e frontend


\subsubsection{Strumenti di sviluppo e deployment}

% è stata adottata la containerizzazione tramite \textit{Docker}, che ha permesso di standardizzare l’ambiente di esecuzione dell’applicazione, riducendo le dipendenze dall’infrastruttura sottostante e facilitando le fasi di distribuzione e avvio del sistema.

