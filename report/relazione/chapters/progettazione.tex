% !TeX spellcheck = it_IT
% !TEX TS-program = pdflatex
% !TEX root = ../main.tex


% ********************************************************************
\section{Progettazione e implementazione}
\label{sec:progettazione}
% ********************************************************************

\subsection{L'obiettivo di Chain4Good}

Chain4Good è una piattaforma decentralizzata di \gls{cf} nata per superare le criticità intrinseche dei sistemi di raccolta fondi tradizionali. 
Il suo obiettivo principale è restituire al donatore un ruolo attivo lungo l’intero ciclo di vita della donazione, mitigando il problema della limitata tracciabilità nell’utilizzo dei fondi tipico dei sistemi centralizzati. \\
A differenza dei modelli tradizionali, nei quali le risorse vengono trasferite integralmente all’Ente beneficiario al termine della raccolta, in Chain4Good l’erogazione dei fondi avviene in maniera incrementale ed è subordinata a un processo di approvazione decentralizzato. In tale contesto, lo sblocco delle risorse è vincolato all’espressione del consenso dei donatori sulle singole richieste di spesa.\\
E'\ importante sottolineare che tale meccanismo non è esente da potenziali comportamenti fraudolenti. La tecnologia \textit{blockchain}, difatti, non è in grado di garantire la veridicità dei dati forniti \textit{off-chain}, quali i preventivi allegati alle richieste di spesa. 
Tuttavia, essa consente di rendere l’intero processo di richiesta, approvazione ed erogazione delle risorse immutabile, trasparente e pubblicamente verificabile, grazie alla registrazione \textit{on-chain} di ogni operazione e di ogni trasferimento di fondi. In questo modo, al donatore è permesso di certificare la congruità tra gli obiettivi dichiarati e quelli effettivamente perseguiti.\\
Chain4Good, dunque, si propone come una piattaforma capace di ridefinire il concetto stesso di donazione, il quale non si configura più come un mero atto di fiducia, bensì come un processo intrinsecamente sicuro e verificabile in ogni sua fase.


%l'idea stessa di donazione evolve

% Ogni operazione di donazione, infatti, viene registrata in modo permanente sulla \textit{blockchain} e può essere verificata pubblicamente, assicurando l'integrità delle informazioni e aumentando il livello di trasparenza. \\
% In tal senso, Chain4Good si distingue dallo stato dell’arte proponendosi come piattaforma orientata non solo alla raccolta fondi per fini prettamente filantropici, ma anche alla partecipazione e fiducia verificabile.

\subsection{Analisi dei requisiti}
In questa sezione sono riportati per punti i requisiti richiesti per il corretto funzionamento della piattaforma.

\subsubsection{Requisiti funzionali}
I requisiti funzionali definiscono le funzionalità che la piattaforma deve implementare. Essi vengono di seguito categorizzati in base agli attori che interagiscono con il sistema. \\

\begin{enumerate}
	\item l'Ente Beneficiario:
	\begin{itemize}
		\item \textbf{Creazione progetto}: l'Ente deve poter avviare una nuova iniziativa di raccolta fondi definendone nome, \textit{budget target} e data di scadenza.
		\item \textbf{Inserimento di una richiesta di spesa}: l'Ente deve poter richiedere il rilascio di una parte dei fondi raccolti avanzando una richiesta di spesa e allegando il relativo preventivo;
		\item \textbf{Caricamento della prova di acquisto}: l'Ente deve poter caricare la fattura che attesti l'effettivo impiego dell'importo richiesto per lo scopo dichiarato;
		\item \textbf{Vincolo di sequenzialità sulle richieste di spesa}: l'Ente non deve poter sottomettere una nuova richiesta di spesa se non ha preventivamente caricato la prova di acquisto relativa alla richiesta precedentemente approvata;
	\end{itemize}
	
	
	\item per i Donatori:
	\begin{itemize}
		\item \textbf{Visualizzazione dei progetti:} il donatore deve poter visualizzare l'elenco delle iniziative di \gls{cf} attive e i relativi dettagli;
		\item \textbf{Donazione ad un progetto}: il donatore deve poter selezionare un progetto e scegliere arbitrariamente l'importo da donare;
		\item \textbf{Votazione delle richieste di spesa}: il donatore deve poter visualizzare il preventivo di spesa allegato dall'Ente ed esprimere una preferenza (se favorevole o contrario);
		\item \textbf{Visualizzazione del portafoglio}: il donatore deve poter visualizzare il saldo disponibile;
	\end{itemize}
	
	
	\item per il Sistema (logica implementata tramite \textit{Smart Contract}):
	\begin{itemize}
		\item \textbf{Registrazione delle donazioni}: il sistema deve registrare \textit{on-chain} ogni donazione effettuata;
		\item \textbf{Blocco dei fondi}: il sistema deve impedire il trasferimento dei fondi, previo consenso dei donatori;
		\item \textbf{Gestione del processo di votazione}: il sistema deve avviare, gestire e concludere il processo di votazione per ogni richiesta di spesa;
		\item \textbf{Erogazione automatica dei fondi}: il sistema deve trasferire automaticamente i fondi al \textit{wallet} dell'Ente, qualora la richiesta di spesa venga approvata dai donatori;
		\item \textbf{Registrazione delle operazioni}: il sistema deve registrare \textit{on-chain} le richieste di spesa, gli esiti delle votazioni e il trasferimento dei fondi sbloccati;
		
	\end{itemize}
	
\end{enumerate}

%il sistema deve consentire esclusivamente agli Enti autorizzati di avviare nuove iniziative di raccolta fondi.

% Descrizione dei requisiti funzionali mediante diagramma UML dei casi d'uso


\subsubsection{Requisiti non funzionali}
Di seguito si riporta l'elenco dei requisiti non funzionali, ossia tutte le caratteristiche che pur non essendo funzionalità, il sistema deve garantire. 

\begin{enumerate}
	
	\item \textbf{Immutabilità}: ogni transazione relativa a donazioni, votazioni e rilascio di fondi deve essere registrata su un registro distribuito in modo permanente e non modificabile;
	
	\item \textbf{Integrità dei dati}: i file pesanti, come preventivi e prove d'acquisto, devono essere memorizzati \textit{off-chain}. Il sistema deve garantire che tali documenti siano riconducibili in modo univoco alle relative operazioni registrate \textit{on-chain}, impedendone la manipolazione;
	
	\item \textbf{Usabilità}: la \textit{Webapp} deve consentire agli utenti di consultare i dati \textit{on-chain}, come lo storico delle donazioni effettuate, attraverso interfacce intuitive;
	
	\item \textbf{Sicurezza}: l'accesso alle funzionalità della piattaforma e alla consultazione dettagliata dei dati deve essere limitato ai soli utenti autenticati;
	
	\item \textbf{Portabilità}: la \textit{Webapp} deve essere fruibile sia da dispositivi \textit{desktop} che \textit{mobile};
\end{enumerate}




%il sistema deve garantire la sicurezza: deve essre accessibile sia da computer che da telefono
% sicurezza: i dati devono rimanere all'interno dll'applicazione, protetti, deve essere sicuro replayact
%forse sicurezza: i dati stanno nell'applicazione e non sono accessibili a chi non è loggato; per vedere i dati devi essere autenticato 
%




     % \item \textbf{Architettura decentralizzata}: il sistema non deve presentare \gls{spof}, per cui anche in caso di down della \textit{Webapp}, i fondi devono rimanere accessibili e gestibili;




\subsection{Architettura del Sistema}
Prima di poter procedere alla progettazione dell’architettura del sistema da realizzare si è resa necessaria l’individuazione delle tecnologie da utilizzare in fase di sviluppo per poter comprendere come queste potessero interagire tra loro e soddisfare tutti i requisiti funzionali e non funzionali emersi dalla precedente fase di analisi.

\subsubsection{Architettura del Software}


\subsubsection*{Frontend}


\subsubsection*{Backend}


\subsubsection*{Blockchain}



\subsubsection{Strumenti di sviluppo}
Per favorire lo sviluppo parallelo e la portabilità del sistema \textit{software} sono stati utilizzati  strumenti riportati in Tabella~\ref{tab:development_tools}.

\smallskip
\begin{table}[h]
	\centering
	\renewcommand{\arraystretch}{1.3}
	\setlength{\tabcolsep}{12pt} % valore di default = 6
	\resizebox{\textwidth}{!}{
		\begin{tabular}{p{4cm} p{9cm}}
			\hline
			\textbf{Strumento} & \textbf{Descrizione} \\
			\hline
			Visual Studio Code & Utilizzato come IDE principale. La scelta è motivata dalla sua natura altamente estensibile, la quale ha consentito l’integrazione di estensioni funzionali allo sviluppo del progetto. \\
			\hline
			Git & \textit{Software} di controllo di versione distribuito, impiegato per la gestione del codice sorgente secondo una strategia di \textit{branching} collaborativa per permettere lo sviluppo parallelo. \\
			\hline
			GitHub & Piattaforma di \textit{hosting} del \textit{repository} remoto, utilizzata per supportare la collaborazione e la condivisione del codice tra i membri del gruppo. \\
			\hline
			Docker & Utilizzato per la standardizzazione e l’isolamento dell’ambiente di esecuzione, garantendo coerenza e riproducibilità tra le diverse fasi di sviluppo e deployment. \\
			\hline
		\end{tabular}
	}
	\caption{Strumenti utilizzati per lo sviluppo del progetto}
	\label{tab:development_tools}
\end{table}





% è stata adottata la containerizzazione tramite \textit{Docker}, che ha permesso di standardizzare l’ambiente di esecuzione dell’applicazione, riducendo le dipendenze dall’infrastruttura sottostante e facilitando le fasi di distribuzione e avvio del sistema.

