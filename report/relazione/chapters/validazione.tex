% !TeX spellcheck = it_IT
% !TEX TS-program = pdflatex
% !TEX root = ../main.tex


% ********************************************************************
\section{Validazione e discussione}
\label{sec:validazione}
% ********************************************************************

Gli obiettivi di progetto sono stati raggiunti con successo. L'applicazione web decentralizzata soddisfa i principali requisiti definiti in fase di analisi, in quanto fornisce un sistema in grado di garantire trasparenza, tracciabilità delle donazioni e coinvolgimento attivo degli utenti.\\
Complessivamente, lo sviluppo della piattaforma è stato condotto ponendo particolare attenzione agli aspetti di sicurezza, affidabilità e qualità del \textit{software}. In particolare, sono state adottate misure preventive volte a mitigare le principali vulnerabilità applicative, quali attacchi di tipo \gls{xss}, SQL injection e \textit{replay attack}, con un focus specifico sui meccanismi di autenticazione. 

Particolare attenzione è stata dedicata alle prestazioni e all’esperienza utente. La gestione della \textit{cache} è stata progettata in modo da garantire un accesso efficiente ai dati, prevedendo meccanismi di invalidazione automatica al superamento di determinate soglie temporali, contribuendo così a una maggiore fluidità dell’applicazione. \\
Analogamente, l’adozione corretta di \textit{framework} orientati al \textit{server-side rendering} ha consentito di migliorare la reattività dell’interfaccia. L’interazione con l’utente è stata progettata, inoltre, per gestire esplicitamente tutti gli stati possibili dell’applicazione, inclusi caricamenti, errori e assenza di risultati, fornendo sempre messaggi informativi chiari.

L'attività di \textit{testing} è stata effettuata progressivamente durante lo sviluppo, piuttosto che sul funzionamento dell'intera piattaforma. In particolare, ogni funzionalità è stata testata immediatamente dopo la sua implementazione attraverso test manuali e mediante l’utilizzo di strumenti di sviluppo dedicati, come \textit{Postman} e richieste HTTP da riga di comando, per validare il corretto comportamento delle API \textit{backend}. Inoltre, sono stati eseguiti test automatizzati sugli \textit{Smart Contract} prima della fase di \textit{deploy}, così da individuare eventuali errori logici o comportamenti inattesi in un ambiente controllato.

Infine, il codice è stato organizzato secondo una struttura modulare e ordinata, con una gestione coerente di file, cartelle, al fine di favorire la manutenibilità e l’evoluzione futura del progetto. Sono state inoltre utilizzate le versioni più recenti e stabili delle librerie adottate.
L’intero sistema è stato concepito per essere facilmente containerizzato mediante \textit{Docker} e \textit{docker-compose}, anche se alcune funzionalità risultano volutamente incomplete o ulteriormente ottimizzabili, in linea con la natura prototipale del progetto.

