% !TeX spellcheck = en_US
% !TEX TS-program = pdflatex
% !TEX root = ../main.tex



% ********************************************************************
\section{Conclusioni e sviluppi futuri}
\label{sec:conclusioni}
% ********************************************************************


% Sviluppi futuri:
% fare in modo che i soldi per le donazioni possano essere spesi presso enti covnenzioanti in modo da tracciare l'effettivo percorso dei soldi donati - tipo 18app - perché uno può comunque allegare 

% è possibile estendere l'utilizzo non solo a enti no-profit ma anche a singoli i quali però possono aprire raccolte fondi tramite enti autorizzati

% utilizzare un iPFS piuttosto che un databse centralizzato (Integrità dei dati off-chain): poiché la blockchain non può ospitare file pesanti, l'integrità di preventivi e prove d'acquisto deve essere garantita tramite il salvataggio degli hash crittografici (SHA-256) on-chain, collegati ai file residenti su sistemi decentralizzati come IPFS.

%la nostra non è completamente decentralizzata perché prevede un admin che registri l'ente, ma questo è un problema comune ad altri studi esistenti 

% 

%limiti 
% nella nostra piattaforma c'è un admin attraverso cui chi vuole avviare iniziative deve passare per poter registrarsi alla piattaforma perché deve avere un nft; se ce l'ha allora la piattaforma capisce e lo fa loggare come ente. ma questo è una cosa comune, in un altro studio patil et al. c'è sempre un admin che però ha la funzione di approvare i progetti per evitare frodi, oltre a dover approvare le singoel fasi di avanzamento. sebbene i fondi siano gestiti automaticamente da uno Smart Contract, l'input decisionale rimane parzialmente centralizzato. Nonostante l'applicazione si definisca decentralizzata, l'Admin mantiene un ruolo di "garante". Nessun progetto è visibile ai finanziatori (Backers) finché l'Admin non lo approva esplicitamente. L'Admin verifica la legittimità della startup e decide se il progetto può essere "listato" sulla piattaforma. Questo indica chiaramente che, affinché lo Smart Contract rilasci i fondi per un traguardo raggiunto, è necessario che l'Admin invii una transazione di approvazione al contratto stess