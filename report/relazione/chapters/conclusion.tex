% !TeX spellcheck = it_IT
% !TEX TS-program = pdflatex
% !TEX root = ../main.tex



% ********************************************************************
\section{Conclusioni e sviluppi futuri}
\label{sec:conclusioni}
% ********************************************************************

Il presente lavoro ha affrontato la progettazione e lo sviluppo di Chain4Good, una piattaforma di \gls{cf} basata su tecnologia blockchain, concepita per rispondere alle criticità strutturali dei sistemi di raccolta fondi tradizionali, con particolare riferimento al contesto filantropico e al Terzo Settore. \\
L'implementazione realizzata ha consentito di verificare la fattibilità del modello proposto e di dimostrare come l’impiego di questa tecnologia possa costituire un valido strumento per aumentare la trasparenza e la sicurezza dell'intero processo di donazione. 
Tuttavia, in ragione della la sua natura sperimentale è possibile evidenziare diverse possibilità di sviluppo volte a migliorare l’efficacia complessiva della piattaforma.

Un primo ambito sviluppo riguarda la gestione completa del ciclo di vita dei progetti. In particolare, il sistema potrebbe essere esteso per gestire scenari attualmente non implementati, quali la cancellazione di un progetto, la disabilitazione di un ente promotore, la modifica del \textit{budget} target o della scadenza temporale, nonché i casi in cui una campagna non raggiunga l’obiettivo economico prefissato. L’introduzione di tali funzionalità consentirebbe di rendere la piattaforma più robusta e aderente a scenari reali.

Dal punto di vista della trasparenza e del controllo delle spese, un possibile miglioramento potrebbe prevedere l'integrazione di tecniche di intelligenza artificiale per la verifica automatica della documentazione caricata \textit{off-chain}, al fine di individuare potenziali anomalie o incongruenze tra il contenuto di fatture e preventivi caricati dall'Ente e gli obiettivi dichiarati all'avvio dell'iniziativa (nella sezione "Piano utilizzo fondi").

Ulteriori sviluppi riguardano il miglioramento dell’interazione con i donatori. L’introduzione di un sistema di notifiche permetterebbe di informare gli utenti in tempo reale sui cambiamenti di stato dei progetti, sull’apertura di nuove richieste di spesa o sull’esito delle votazioni. Allo stesso modo, la possibilità per gli Enti di pubblicare aggiornamenti periodici, corredati da testi e immagini, consentirebbe di rafforzare il coinvolgimento della comunità e di rendere il processo di donazione più partecipativo e trasparente.

Infine, i risultati ottenuti evidenziano come l’architettura proposta non sia strettamente limitata al dominio applicativo di riferimento. Infatti, sebbene Chain4Good sia stata progettata specificamente per il Terzo Settore, il modello adottato potrebbe essere generalizzato e applicato anche ad altri scenari di raccolta fondi, includendo iniziative promosse da soggetti privati. In questo senso, l’approccio proposto rappresenta una possibile base per lo sviluppo di piattaforme di \gls{cf} più aperte e trasparenti, in grado di estendere i benefici della tecnologia blockchain a una platea di utilizzatori sempre più vasta.
