% !TeX spellcheck = it_IT
% !TEX TS-program = pdflatex
% !TEX root = ../main.tex



% ********************************************************************
\section{Conclusioni e sviluppi futuri}
\label{sec:conclusioni}
% ********************************************************************

Il presente lavoro ha affrontato la progettazione e lo sviluppo di Chain4Good, una piattaforma di \gls{cf} basata su tecnologia blockchain, con particolare riferimento al contesto filantropico e al Terzo Settore.
Nonostante la natura prototipale, la sua implementazione ha consentito di verificare la fattibilità del modello proposto e di dimostrare come l’impiego di questa tecnologia possa costituire un valido strumento per aumentare la trasparenza e la sicurezza dell'intero processo di donazione. 

Tuttavia, dall’analisi critica del sistema sono emerse alcune possibili evoluzioni funzionali, volte a migliorare l’efficacia complessiva della piattaforma e ad ampliarne le potenzialità applicative.

Un primo ambito di sviluppo individuato riguarda la partecipazione dei donatori alle votazioni sulle richieste di spesa. Poiché l’espressione del voto comporta il sostenimento di un costo di transazione, alcuni utenti potrebbero scegliere di non partecipare attivamente al processo decisionale (confidando nel fatto che l’esito venga comunque determinato da altri partecipanti o dal meccanismo di approvazione automatica). In prospettiva, l’introduzione di meccanismi incentivanti potrebbe contribuire ad aumentare il coinvolgimento degli utenti e a ridurre l’impatto dei costi di transazione.

Un ulteriore sviluppo riguarda la gestione completa del ciclo di vita dei progetti. In particolare, il sistema potrebbe essere esteso per gestire scenari attualmente non implementati, quali la cancellazione di un progetto, la disabilitazione di un ente promotore, la modifica del \textit{budget} target o della scadenza temporale, nonché i casi in cui una campagna non raggiunga l’obiettivo economico prefissato. L’introduzione di tali funzionalità consentirebbe di rendere la piattaforma più robusta e aderente a scenari reali.

Dal punto di vista della trasparenza e del controllo delle spese, un possibile miglioramento potrebbe prevedere l'integrazione di tecniche di intelligenza artificiale per la verifica automatica della documentazione caricata \textit{off-chain}, al fine di individuare potenziali anomalie o incongruenze tra il contenuto di fatture e preventivi e quanto dichiarato in fase di creazione del progetto (nella sezione "Piano utilizzo fondi").

Ulteriori sviluppi riguardano il miglioramento dell’interazione con i donatori. L’introduzione di un sistema di notifiche permetterebbe di informare gli utenti in tempo reale sui cambiamenti di stato dei progetti, sull’apertura di nuove richieste di spesa o sull’esito delle votazioni. Allo stesso modo, la possibilità per gli enti di pubblicare aggiornamenti periodici, corredati da testi e immagini, consentirebbe di rafforzare il coinvolgimento della comunità e di rendere il processo di donazione più partecipativo e trasparente.

Infine, i risultati ottenuti evidenziano inoltre come l’architettura proposta non sia strettamente limitata al dominio applicativo di riferimento. Infatti, sebbene Chain4Good sia stata progettata specificamente per il Terzo Settore, il modello proposto potrebbe essere esteso anche ad altri contesti applicativi. Questa evoluzione ha il potenziale di democratizzare ulteriormente l'accesso al \gls{cf}, estendendo i vantaggi della blockchain a una platea di utilizzatori sempre più vasta.


% Infine, un altro rischio riconducibile al dominio applicativo, riguarda la possibile scarsa partecipazione degli utenti alle votazioni sulle richieste di spesa. Poiché l’espressione del voto comporta il sostenimento di un costo di transazione (\textit{gas fee}), è plausibile che alcuni donatori scelgano di non partecipare attivamente al processo decisionale, confidando nel fatto che l’esito venga comunque determinato da altri partecipanti o dal meccanismo di approvazione automatica. Nel contesto del presente lavoro, tale rischio non è stato mitigato mediante strategie specifiche, in quanto la progettazione di meccanismi incentivanti avrebbe introdotto un livello di complessità non coerente con la natura prototipale del sistema.



% Sviluppi futuri:
% fare in modo che i soldi per le donazioni possano essere spesi presso enti covnenzioanti in modo da tracciare l'effettivo percorso dei soldi donati - tipo 18app - perché uno può comunque allegare 

% è possibile estendere l'utilizzo non solo a enti no-profit ma anche a singoli i quali però possono aprire raccolte fondi tramite enti autorizzati

%la nostra non è completamente decentralizzata perché prevede un admin che registri l'ente, ma questo è un problema comune ad altri studi esistenti 


%limiti 
% nella nostra piattaforma c'è un admin attraverso cui chi vuole avviare iniziative deve passare per poter registrarsi alla piattaforma perché deve avere un nft; se ce l'ha allora la piattaforma capisce e lo fa loggare come ente. ma questo è una cosa comune, in un altro studio patil et al. c'è sempre un admin che però ha la funzione di approvare i progetti per evitare frodi, oltre a dover approvare le singoel fasi di avanzamento. sebbene i fondi siano gestiti automaticamente da uno Smart Contract, l'input decisionale rimane parzialmente centralizzato. Nonostante l'applicazione si definisca decentralizzata, l'Admin mantiene un ruolo di "garante". Nessun progetto è visibile ai finanziatori (Backers) finché l'Admin non lo approva esplicitamente. L'Admin verifica la legittimità della startup e decide se il progetto può essere "listato" sulla piattaforma. Questo indica chiaramente che, affinché lo Smart Contract rilasci i fondi per un traguardo raggiunto, è necessario che l'Admin invii una transazione di approvazione al contratto stess