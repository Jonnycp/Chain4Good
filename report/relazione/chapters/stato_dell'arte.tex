% !TeX spellcheck = it_IT
% !TEX TS-program = pdflatex
% !TEX root = ../main.tex

% ********************************************************************
\section{Stato dell'arte}
\label{sec:stato dell'arte}
% ********************************************************************

\subsection{Crowdfunding}

Il \textit{\gls{cf} }è un modello di finanziamento collettivo in cui una pluralità di individui decide di destinare il proprio denaro, prevalentemente tramite piattaforme digitali, a supporto di progetti e iniziative di varia natura \cite{brunello2016crowdfunding}. \\
In ragione della sua etimologia, dall'inglese \textit{crowd} "folla" e \textit{funding}, finanziamento, il \gls{cf} è stato definito come una pratica di microfinanziamento "dal basso" \cite{thalassinos2023crowdfunding}, la cui peculiarità risiede nella capacità di aggregare numerosi contributi finanziari di modesta entità a partire da un'ampia platea di sostenitori. 

La letteratura attribuisce al Web 2.0 il principale catalizzatore del successo del \gls{cf}. Lo sviluppo di Internet e la capillare diffusione di canali digitali, come i \textit{social-media}, infatti, ha permesso non solo di ampliare la platea di donatori, ma anche di abbattere i limiti geografici, trasformando la "folla" in una comunità attiva e globale.

%Questo modello di finanziamento si è progressivamente evoluto in forme differenti in funzione della natura del contributo richiesto e delle aspettative di ritorno dei finanziatori: 
%\begin{itemize}
	%\item \gls{dcf}: i contributi economici sono erogati senza alcuna aspettativa di ritorno materiale o finanziario. La donazione è motivata esclusivamente dal desiderio di sostenere una causa di interesse collettivo o di pubblica utilità \cite{salido2021mapping};
	%\item \gls{rbc}: \textit{keep-it-all} oppue \textit{all-or-nothing}
	%\item \gls{ecf}: 
	%\item \gls{lcf}:
%\end{itemize} 




\subsubsection{Donation-based crowdfunding}
Il \gls{dcf} ha attirato una crescente attenzione accademica configurandosi come uno strumento rivoluzionario per il Terzo Settore.



%rapida espansione con le classiche piattaforme
%limiti
%blockchain 
%lavori su blockchain e crowdfunding



% come una modalità alternativa di raccolta fondi, capace di sostenere un’ampia gamma di iniziative e, in particolare, di rappresentare una potenziale fonte di finanziamento per le organizzazioni non profit impegnate in attività di pubblico interesse.




\subsection{Crowdfunding e blockchain}



%In ragione della sua etimologia, dall'inglese \textit{crowd} "folla" e \textit{funding}, finanziamento, il \textit{crowdfunding} è stato definito come una pratica di microfinanziamento dal basso \cite{thalassinos2023crowdfunding}, capace di mobilitare persone e risorse.


%Il \textit{crowdfunding} benefico (o \textit{donation-based crowdfunding}) rappresenta una forma specifica di raccolta fondi, in cui i contributi economici sono erogati come donazioni, senza alcuna aspettativa di ritorno finanziario \cite{baber2020blockchain}.


%Tale modello è ampiamente utilizzato per sostenere iniziative sociali, umanitarie, sanitarie o culturali, promosse da enti non profit 

%Il problema delle piattaforme esistenti è che una volta effettuata la donazione, il donatore perde il controllo sul destino del proprio contributo. Questa asimmetria informativa rappresenta una delle principali cause di sfiducia





%Risulta evidente che il crowdfunding rappresenti uno strumento fondamentale per la democratizzazione dell’accessibilità ai finanziamenti per una vasta gamma di progetti e organizzazioni; tuttavia, i limiti caratteristici delle piattaforme tradizionali hanno evidenziato la necessità di un’evoluzione verso modelli Web3 – Kickstarter stessa ne è ha preso atto in un comunicato del 2021

% il nostro progetto riguarda la raccolta fondi da parte di enti benefici mentre in letteratura ecc. 

% La blockchain, in quanto registro distribuito, immutabile e verificabile consente di memorizzare transazioni in modo trasparente e resistente alle manomissioni, eliminando la necessità di un’autorità centrale di fiducia

% Kickstarter - ogni progetto ha un obiettivo economico. se non raggiunto le donazioni credo sianor estituite 