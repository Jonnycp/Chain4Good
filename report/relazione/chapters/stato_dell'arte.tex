% !TeX spellcheck = it_IT
% !TEX TS-program = pdflatex
% !TEX root = ../main.tex

% ********************************************************************
\section{Stato dell'arte}
\label{sec:stato dell'arte}
% ********************************************************************

\subsection{Crowdfunding}

Il \textit{\gls{cf}} è un modello di finanziamento collettivo in cui una pluralità di individui decide di destinare il proprio denaro, prevalentemente tramite piattaforme digitali, a supporto di progetti e iniziative di varia natura \cite{brunello2016crowdfunding}. \\
In ragione della sua etimologia, dall'inglese \textit{crowd} "folla" e \textit{funding}, finanziamento, il \gls{cf} è stato definito come una pratica di microfinanziamento "dal basso" \cite{thalassinos2023crowdfunding}, la cui peculiarità risiede nella capacità di aggregare numerosi contributi finanziari di modesta entità a partire da un'ampia platea di sostenitori. 

La letteratura attribuisce al Web 2.0 il principale catalizzatore del successo del \gls{cf} \cite{brunetti2016web}. Lo sviluppo di Internet e la capillare diffusione di canali digitali di comunicazione, come i \textit{social-media}, infatti, ha permesso non solo di ampliare la platea di donatori, ma anche di abbattere i limiti geografici, trasformando la "folla" in una comunità attiva e globale. \\
Inoltre, la nascita di infrastrutture digitali dedicate, come \textit{Kickstarter} e \textit{GoFundme}, è stato determinate per garantire la scalabilità e la sicurezza del fenomeno. 

In questo scenario, il modello contemporaneo di \gls{cf} si articola in un’architettura tripartita, che vede l'interazione sinergica di tre attori chiave: il promotore dell'iniziativa, i sostenitori e la piattaforma digitale \cite{alia2024ihsan}. Quest’ultima non funge da mera vetrina, ma rappresenta l’infrastruttura tecnologica che media le interazioni tra le parti, facilitando il processo di raccolta fondi, la diffusione delle informazioni e il coordinamento delle attività connesse alla realizzazione del progetto. Sebbene la struttura relazionale del \gls{cf} rimanga invariata, la natura del contributo richiesto e le aspettative di ritorno dei sostenitori, rappresentano gli elementi chiave che ne definiscono la tassonomia. E'\ sulla base di questi criteri, infatti, che gli studi convergono nel classificare le seguenti tipologie di \gls{cf}:
\begin{itemize}
	\item \gls{dcf}: i contributi economici sono erogati senza alcuna aspettativa di ritorno materiale o finanziario. La donazione è motivata esclusivamente dal desiderio di sostenere una causa di interesse collettivo o di pubblica utilità; per questa ragione, la \gls{dcf} è stata definita come la forma più "pura" di \textit{crowdfunding} \cite{salido2021mapping};
	\item \gls{rbc}: i sostenitori finanziano un progetto in cambio di una ricompensa, generalmente di natura non finanziaria (come riconoscimenti simbolici oppure ricompense tangibili, configurandosi spesso come un vero e proprio "pre-ordine" del prodotto) \cite{hohen2025reward};
	\item \gls{ecf}: il finanziatore, sia esso un individuo o un ente, riceve quote societarie o titoli partecipativi dell'azienda, in cambio del capitale investito \cite{kuti2017equity}
	\item \gls{lcf}: noto anche come \textit{debt-based crowdfunding}, prevede che il capitale versato dai sostenitori venga rimborsato dal promotore entro una scadenza prestabilita, comprensivo di un tasso di interesse pattuito \cite{hossain2017crowdfunding};
\end{itemize} 

E'\ importante sottolineare che, nonostante la natura prettamente finanziaria degli ultimi due modelli, l'elemento che li riconduce univocamente al paradigma del \textit{crowdfunding} è la modalità di raccolta: il capitale non è più appannaggio di un singolo grande istituto di credito, ma deriva dalla somma di innumerevoli micro-investimenti operati da una moltitudine di individui. È proprio questo a sancire la natura "dal basso" di tali strumenti, trasformando ogni cittadino in un potenziale nodo di una rete di finanziamento globale e democratico.


\subsubsection{DCF e Terzo Settore}
Il \gls{dcf} ha assunto un rilievo strategico per il Terzo settore, configurandosi come uno strumento capace di potenziare le attività delle Organizzazioni \textit{no-profit}.

%Come per il \gls{rbc}, è possibile adoperare un modello di raccolta del tipo: \textit{keep-it-all}, per cui le quote versate tornano ai donatori se non è raggiunto il 100\% del \textit{budget} richiesto, oppure \textit{all-or-nothing}, per cui il proponente trattiene i fondi raccolti anche qualora l'obiettivo non venga raggiunto




\subsection{Crowdfunding e blockchain}





%Tale modello è ampiamente utilizzato per sostenere iniziative sociali, umanitarie, sanitarie o culturali, promosse da enti non profit 

%Il problema delle piattaforme esistenti è che una volta effettuata la donazione, il donatore perde il controllo sul destino del proprio contributo. Questa asimmetria informativa rappresenta una delle principali cause di sfiducia





%Risulta evidente che il crowdfunding rappresenti uno strumento fondamentale per la democratizzazione dell’accessibilità ai finanziamenti per una vasta gamma di progetti e organizzazioni; tuttavia, i limiti caratteristici delle piattaforme tradizionali hanno evidenziato la necessità di un’evoluzione verso modelli Web3 – Kickstarter stessa ne è ha preso atto in un comunicato del 2021

% La blockchain, in quanto registro distribuito, immutabile e verificabile consente di memorizzare transazioni in modo trasparente e resistente alle manomissioni, eliminando la necessità di un’autorità centrale di fiducia

% Kickstarter - ogni progetto ha un obiettivo economico. se non raggiunto le donazioni credo siano restituite 