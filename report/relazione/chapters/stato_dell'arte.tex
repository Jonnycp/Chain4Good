% !TeX spellcheck = it_IT
% !TEX TS-program = pdflatex
% !TEX root = ../main.tex

% ********************************************************************
\section{Stato dell'arte}
\label{sec:stato dell'arte}
% ********************************************************************

\subsection{Crowdfunding e Blockchain}

Il \textit{crowdfunding} è un modello di finanziamento collettivo in cui più persone decidono di stanziare in maniera volontaria il proprio denaro, prevalentemente tramite piattaforme digitali, a supporto di iniziative di varia natura (cause sociali, \textit{start-up}) \cite{brunello2016crowdfunding}. In ragione della sua etimologia, dall'inglese \textit{crowd} "folla" e \textit{funding}, finanziamento, il crowdfunding è stata definita come una pratica di microfinanziamento dal basso \cite{thalassinos2023crowdfunding}, capace di mobilitare persone e risorse.
A differenza del \textit{fundraising}, 


Se i contributi economici sono erogati come donazione, senza alcuna aspettativa di ritorno (sia finanziario che no), allora si parla di \gls{dcf}. In questo modello, la donazione è motivata esclusivamente dal desiderio di sostenere una causa di interesse collettivo o di pubblica utilità \cite{salido2021mapping}.




%Il \textit{crowdfunding} benefico (o \textit{donation-based crowdfunding}) rappresenta una forma specifica di raccolta fondi, in cui i contributi economici sono erogati come donazioni, senza alcuna aspettativa di ritorno finanziario \cite{baber2020blockchain}.


%Tale modello è ampiamente utilizzato per sostenere iniziative sociali, umanitarie, sanitarie o culturali, promosse da enti non profit 

%Il problema delle piattaforme esistenti è che una volta effettuata la donazione, il donatore perde il controllo sul destino del proprio contributo. Questa asimmetria informativa rappresenta una delle principali cause di sfiducia





%Risulta evidente che il crowdfunding rappresenti uno strumento fondamentale per la democratizzazione dell’accessibilità ai finanziamenti per una vasta gamma di progetti e organizzazioni; tuttavia, i limiti caratteristici delle piattaforme tradizionali hanno evidenziato la necessità di un’evoluzione verso modelli Web3 – Kickstarter stessa ne è ha preso atto in un comunicato del 2021

% il nostro progetto riguarda la raccolta fondi da parte di enti benefici mentre in letteratura ecc. 

% La blockchain, in quanto registro distribuito, immutabile e verificabile consente di memorizzare transazioni in modo trasparente e resistente alle manomissioni, eliminando la necessità di un’autorità centrale di fiducia

% Kickstarter - ogni progetto ha un obiettivo economico. se non raggiunto le donazioni credo sianor estituite 