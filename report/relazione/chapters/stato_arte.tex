% !TeX spellcheck = it_IT
% !TEX TS-program = pdflatex
% !TEX root = ../main.tex



% ********************************************************************
\section{Stato dell'arte}
\label{sec:stato arte}
% ********************************************************************

\subsection{Crowdfunding}

Il \textit{\gls{cf}} è un modello di finanziamento collettivo in cui una pluralità di individui decide di destinare il proprio denaro, prevalentemente tramite piattaforme digitali, a supporto di progetti e iniziative di varia natura \cite{brunello2016crowdfunding}. \\
In ragione della sua etimologia, dall'inglese \textit{crowd} "folla" e \textit{funding}, finanziamento, il \gls{cf} è stato definito come una pratica di microfinanziamento "dal basso" \cite{thalassinos2023crowdfunding}, la cui peculiarità risiede nella capacità di aggregare numerosi contributi finanziari di modesta entità a partire da un'ampia platea di sostenitori. 

La letteratura attribuisce al Web 2.0 il principale catalizzatore del successo del \gls{cf} \cite{brunetti2016web}. Lo sviluppo di Internet e la capillare diffusione di canali digitali di comunicazione, come i \textit{social-media}, infatti, ha permesso non solo di ampliare la platea di donatori, ma anche di abbattere i limiti geografici, trasformando la "folla" in una comunità attiva e globale. \\
Inoltre, la nascita di infrastrutture digitali dedicate, come \textit{Kickstarter} e \textit{GoFundme}, è stato determinate per garantire la scalabilità del fenomeno. 

In questo scenario, il modello contemporaneo di \gls{cf} si articola in un’architettura tripartita, che vede l'interazione sinergica di tre attori chiave: il promotore dell'iniziativa, i sostenitori e la piattaforma digitale \cite{alia2024ihsan}. Quest’ultima non funge da mera vetrina, ma rappresenta l’infrastruttura tecnologica che media le interazioni tra le parti, facilitando il processo di raccolta fondi, la diffusione delle informazioni e il coordinamento delle attività connesse alla realizzazione del progetto. Sebbene la struttura relazionale del \gls{cf} rimanga invariata, la natura del contributo richiesto e le aspettative di ritorno dei sostenitori, rappresentano gli elementi chiave che ne definiscono la tassonomia. E'\ sulla base di questi criteri, infatti, che gli studi convergono nel classificare le seguenti tipologie di \gls{cf}:
\begin{itemize}
	\item \gls{dcf}: i contributi economici sono erogati senza alcuna aspettativa di ritorno materiale o finanziario. La donazione è motivata esclusivamente dal desiderio di sostenere una causa di interesse collettivo o di pubblica utilità; per questa ragione, la \gls{dcf} è stata definita come la forma più "pura" di \textit{crowdfunding} \cite{salido2021mapping};
	\item \gls{rbc}: i sostenitori finanziano un progetto in cambio di una ricompensa, generalmente di natura non finanziaria (come riconoscimenti simbolici oppure ricompense tangibili, configurandosi spesso come un vero e proprio "pre-ordine" del prodotto) \cite{hohen2025reward};
	\item \gls{ecf}: il finanziatore, sia esso un individuo o un ente, riceve quote societarie o titoli partecipativi dell'azienda, in cambio del capitale investito \cite{kuti2017equity}
	\item \gls{lcf}: noto anche come \textit{debt-based crowdfunding}, prevede che il capitale versato dai sostenitori venga rimborsato dal promotore entro una scadenza prestabilita, comprensivo di un tasso di interesse pattuito \cite{hossain2017crowdfunding};
\end{itemize} 

Nonostante la natura prettamente finanziaria degli ultimi due modelli, l'elemento che li riconduce univocamente al paradigma del \textit{crowdfunding} è la modalità di raccolta: il capitale non è più appannaggio di un singolo grande istituto di credito, ma deriva dalla somma di innumerevoli micro-investimenti operati da una moltitudine di individui. E'\ proprio questo a sancire la natura "dal basso" di tali strumenti, trasformando ogni cittadino in un potenziale nodo di una rete di finanziamento globale e democratico. 


\subsubsection{Limiti delle piattaforme tradizionali di CF}
La letteratura converge nel considerare le piattaforme di \gls{cf} caratterizzate da una serie di criticità strutturali, riconducibili principalmente al loro modello architetturale. Si distinguono:

\begin{itemize}
	\item \textbf{mancanza di trasparenza nell'utilizzo dei fondi}: i donatori non dispongono di strumenti efficaci per verificare l'intero ciclo di vita delle donazioni \cite{alia2024ihsan}. La tracciabilità delle quote donate è, infatti, demandata al fruitore della donazione, il quale ha il compito di fornire aggiornamenti sullo stato di avanzamento dell'iniziativa finanziata;
	
	\item \textbf{scarsa fiducia e frodi}: l’assenza di meccanismi di verifica e controllo automatizzati rende le piattaforme di \gls{cf} tradizionali vulnerabili a comportamenti fraudolenti, quali la creazione di campagne ingannevoli o la mancata realizzazione dei progetti finanziati. Tale criticità accentua l’asimmetria informativa tra promotori e donatori, alimentando una diffusa sfiducia nei confronti delle piattaforme \cite{rejeb2025mapping}.
	
	\item \textbf{centralizzazione}: l'architettura centralizzata utilizzata dalla maggior parte delle piattaforme (come GoFundMe) introduce \gls{spof} e attribuisce la gestione dei fondi raccolti interamente alla piattaforma \cite{mukherjee2024crowdfunding};
	
	\item \textbf{mancanza di sicurezza}: la gestione centralizzata dei dati espone le piattaforme ad attacchi malevoli, con conseguenze rilevanti in termini di perdita di fondi e fiducia degli utenti \cite{mukherjee2024crowdfunding}. 
	
	% \item \textbf{alti costi operativi}: le piattaforme tradizionali applicano commissioni elevate per la gestione delle campagne, il processamento dei pagamenti e l’intermediazione finanziaria, riducendo il capitale disponibile per i progetti finanziati 
	
	
	%le piattaforme tradizionali impongono commissioni significative () che riducono il capitale raccolto per progetto;
	
	
	%la percezione di sfiducia nei confronti della piattaforma o dei promotori delle campagne può incidere negativamente sulla scelta di donare \cite{ferreira2022unpeel};

\end{itemize}

Nel complesso, queste criticità evidenziano come le piattaforme tradizionali di \gls{cf} si fondino su un modello fortemente fiduciario, nel quale il corretto funzionamento del sistema dipende dal comportamento onesto degli intermediari e dei promotori delle iniziative. 
Sebbene questo paradigma abbia dunque democratizzato l'accesso al capitale, l'architettura adottata introduce inefficienze strutturali che limitano il potenziale del modello di \textit{crowdfunding}.



\subsection{Crowdfunding e blockchain}



%cite shelke
%Alcuni ricercatori studiano la possibilità di integrare la blockchain nelle piattaforme esistenti tipo per garantire maggiore trasparenza nella gestione dei fondi oppure per 8min


%https://link.springer.com/chapter/10.1007/978-981-15-1137-0_6#Sec7
%In un processo di crowdfunding esistente, sono tre le parti attivamente coinvolte: finanziatori, fundraiser e piattaforma di crowdfunding. La quarta parte è una banca, presso la quale viene depositato il denaro, che agisce come membro passivo. Nel crowdfunding basato su blockchain, la banca verrà sostituita dal portafoglio di criptovalute. 
%La maggior parte delle piattaforme di crowdfunding adotta due opzioni di finanziamento: "tutto o niente" o "tieni tutto". Il modello di crowdfunding più utilizzato è il modello "tutto o niente". Questo modello si basa sull'idea che un progetto possa essere avviato solo al raggiungimento dell'obiettivo di investimento, altrimenti i fondi verranno restituiti ai finanziatori. 




%Nel contesto del \gls{cf}, l’interesse verso la blockchain deriva dalla possibilità di affrontare alcune delle criticità associate ai modelli di piattaforma tradizionali. In particolare, la natura decentralizzata della blockchain consente di ridurre il ruolo dell’intermediario fiduciario, mentre l’immutabilità dei dati registrati garantisce l’integrità delle informazioni relative alle transazioni e agli eventi associati alle campagne di raccolta fondi. Ogni operazione può essere verificata pubblicamente, aumentando il livello di trasparenza nei confronti di tutti gli attori coinvolti.


%Come per il \gls{rbc}, è possibile adoperare un modello di raccolta del tipo: \textit{keep-it-all}, per cui le quote versate tornano ai donatori se non è raggiunto il 100\% del \textit{budget} richiesto, oppure \textit{all-or-nothing}, per cui il proponente trattiene i fondi raccolti anche qualora l'obiettivo non venga raggiunto


%Risulta evidente che il crowdfunding rappresenti uno strumento fondamentale per la democratizzazione dell’accessibilità ai finanziamenti per una vasta gamma di progetti e organizzazioni; tuttavia, i limiti caratteristici delle piattaforme tradizionali hanno evidenziato la necessità di un’evoluzione verso modelli Web3 – Kickstarter stessa ne è ha preso atto in un comunicato del 2021
