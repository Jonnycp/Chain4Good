% !TeX spellcheck = it_IT
% !TEX TS-program = pdflatex
% !TEX root = ../main.tex



% ********************************************************************
\section{Stato dell'arte}
\label{sec:stato arte}
% ********************************************************************

\subsection{Crowdfunding}

Il \textit{\gls{cf}} è un modello di finanziamento collettivo in cui una pluralità di individui decide di destinare il proprio denaro, prevalentemente tramite piattaforme digitali, a supporto di progetti e iniziative di varia natura \cite{brunello2016crowdfunding}. \\
In ragione della sua etimologia, dall'inglese \textit{crowd} "folla" e \textit{funding}, finanziamento, il \gls{cf} è stato definito come una pratica di microfinanziamento "dal basso" \cite{thalassinos2023crowdfunding}, la cui peculiarità risiede nella capacità di aggregare numerosi contributi finanziari di modesta entità a partire da un'ampia platea di sostenitori. 

La letteratura attribuisce al Web 2.0 il principale catalizzatore del successo del \gls{cf} \cite{brunetti2016web}. Lo sviluppo di Internet e la capillare diffusione di canali digitali di comunicazione, come i \textit{social-media}, infatti, ha permesso non solo di ampliare la platea di donatori, ma anche di abbattere i limiti geografici, trasformando la "folla" in una comunità attiva e globale. \\
Inoltre, la nascita di infrastrutture digitali dedicate, come \textit{Kickstarter} e \textit{GoFundme}, è stato determinate per garantire la scalabilità del fenomeno. 

In questo scenario, il modello contemporaneo di \gls{cf} si articola in un’architettura tripartita, che vede l'interazione sinergica di tre attori chiave: il promotore dell'iniziativa, i sostenitori e la piattaforma digitale \cite{alia2024ihsan}. Quest’ultima non funge da mera vetrina, ma rappresenta l’infrastruttura tecnologica che media le interazioni tra le parti, facilitando il processo di raccolta fondi, la diffusione delle informazioni e il coordinamento delle attività connesse alla realizzazione del progetto. Sebbene la struttura relazionale del \gls{cf} rimanga invariata, la natura del contributo richiesto e le aspettative di ritorno dei sostenitori, rappresentano gli elementi chiave che ne definiscono la tassonomia. E'\ sulla base di questi criteri, infatti, che gli studi convergono nel classificare le seguenti tipologie di \gls{cf}:
\begin{itemize}
	\item \gls{dcf}: i contributi economici sono erogati senza alcuna aspettativa di ritorno materiale o finanziario. La donazione è motivata esclusivamente dal desiderio di sostenere una causa di interesse collettivo o di pubblica utilità; per questa ragione, la \gls{dcf} è stata definita come la forma più "pura" di \textit{crowdfunding} \cite{salido2021mapping};
	\item \gls{rbc}: i sostenitori finanziano un progetto in cambio di una ricompensa, generalmente di natura non finanziaria (come riconoscimenti simbolici oppure ricompense tangibili, configurandosi spesso come un vero e proprio "pre-ordine" del prodotto) \cite{hohen2025reward};
	\item \gls{ecf}: il finanziatore, sia esso un individuo o un ente, riceve quote societarie o titoli partecipativi dell'azienda, in cambio del capitale investito \cite{kuti2017equity}
	\item \gls{lcf}: noto anche come \textit{debt-based crowdfunding}, prevede che il capitale versato dai sostenitori venga rimborsato dal promotore entro una scadenza prestabilita, comprensivo di un tasso di interesse pattuito \cite{hossain2017crowdfunding};
\end{itemize} 

% Nonostante la natura prettamente finanziaria degli ultimi due modelli, l'elemento che li riconduce univocamente al paradigma del \textit{crowdfunding} è la modalità di raccolta: il capitale non è più appannaggio di un singolo grande istituto di credito, ma deriva dalla somma di innumerevoli micro-investimenti operati da una moltitudine di individui. E'\ proprio questo a sancire la natura "dal basso" di tali strumenti, trasformando ogni cittadino in un potenziale nodo di una rete di finanziamento globale e democratico. 


\subsubsection{Limiti delle piattaforme tradizionali di CF}
La letteratura converge nel considerare le piattaforme di \gls{cf} caratterizzate da una serie di criticità strutturali, riconducibili principalmente al loro modello architetturale centralizzato. Si distinguono:

\begin{itemize}
	\item limitata trasparenza nell'utilizzo dei fondi: i donatori non dispongono di strumenti efficaci per verificare l'intero ciclo di vita delle donazioni \cite{alia2024ihsan}. La tracciabilità delle quote donate è, infatti, demandata al fruitore della donazione, il quale ha il compito di fornire aggiornamenti sullo stato di avanzamento dell'iniziativa finanziata;
	
	\item scarsa fiducia e rischio di frodi: l'assenza di protocolli di verifica automatizzati rende le piattaforme di \gls{cf} tradizionali vulnerabili a comportamenti fraudolenti, come la creazione di campagne ingannevoli o la mancata realizzazione dei progetti finanziati \cite{rejeb2025mapping}. Questo clima di incertezza scoraggia la partecipazione degli utenti, alimentando una diffusa sfiducia nei confronti delle piattaforme.
	
	\item centralizzazione: l'architettura centralizzata utilizzata dalla maggior parte delle piattaforme introduce \gls{spof} e attribuisce la gestione dei fondi raccolti interamente alla piattaforma \cite{mukherjee2024crowdfunding};
	
	\item mancanza di sicurezza: la gestione centralizzata dei dati espone le piattaforme ad attacchi malevoli, con conseguenze rilevanti in termini di perdita di fondi e fiducia degli utenti \cite{mukherjee2024crowdfunding}. 
	
	% \item alti costi operativi: le piattaforme tradizionali applicano commissioni elevate per la gestione delle campagne, riducendo il capitale disponibile per i progetti finanziati;
	

\end{itemize}

Nel complesso, queste criticità evidenziano come le piattaforme tradizionali di \gls{cf} si fondino su un modello fortemente fiduciario, nel quale il corretto funzionamento del sistema dipende dal comportamento onesto degli intermediari e dei promotori delle iniziative. 
Sebbene questo paradigma abbia dunque democratizzato l'accesso al capitale, l'architettura adottata introduce inefficienze strutturali che limitano il potenziale del modello di \gls{cf}.



\subsection{Crowdfunding e blockchain}

\subsubsection{Il ruolo della blockchain nel \textit{crowdfunding}}
La letteratura converge nel considerare la \textit{blockchain} come una soluzione promettente per il superamento delle criticità strutturali delle piattaforme di \gls{cf} esistenti \cite{shelke2022blockchain}. L'efficacia di tale tecnologia risiede nelle proprietà native di decentralizzazione, trasparenza, immutabilità e sicurezza (Tabella~\ref{tab:blockchain_vantaggi}) che implementa. 

La \textit{blockchain}, infatti, si configura come un registro distribuito, condiviso e immutabile, mantenuto da una rete di nodi interconnessi secondo un’architettura \gls{p2p}. Essa consiste in una catena sequenziale di blocchi legati tra loro da meccanismi crittografici. Ciascun blocco contiene un insieme di transazioni (operazioni atomiche come il trasferimento di \textit{asset} digitali), la cui integrità è preservata da protocolli di consenso eseguiti in modo distribuito dalla rete \cite{nakamoto2007bitcoin}.

L'integrazione di tale architettura nel \gls{cf} è motivata dai seguenti vantaggi:
\begin{itemize}
	\item la natura distribuita del registro (proprietà di decentralizzazione) permette di eliminare i \gls{spof}, tipici delle architetture centralizzate, e di attribuire ad un insieme di nodi piuttosto che al singolo ente, la gestione dei fondi raccolti;
	\item la proprietà di immutabilità assicura l’integrità delle informazioni memorizzate \textit{on-chain} \cite{salido2021mapping}, come le transazioni di donazione o i dettagli sulle campagne di raccolta fondi;
	\item la trasparenza intrinseca del \textit{ledger}, permette ai donatori di monitorare l’intero ciclo di vita delle donazioni, rafforzando in questo modo la fiducia nelle piattaforme.
\end{itemize}

% L’impiego di tale tecnologia nel \gls{cf}, dunque, risponde a specifiche esigenze di affidabilità, sicurezza e tracciabilità delle donazioni.

\medskip
\begin{table}[h]
\centering
\renewcommand{\arraystretch}{1.3} % spazio verticale tra le righe
\resizebox{\textwidth}{!}{
	\begin{tabular}{p{3.5cm} p{10cm}}
		\hline
		\textbf{Proprietà} & \textbf{Descrizione} \\
		\hline
		Decentralizzazione & La replicazione del registro su più nodi consente al sistema di operare correttamente anche in presenza di guasti, eliminando i \gls{spof} \cite{rodrigues2010peer}. \\
		\hline
		Immutabilità & Ogni transazione è irreversibile; una volta registrata nella blockchain, non può essere cancellata o modificata, poiché qualsiasi alterazione comprometterebbe l’intera catena dei blocchi \cite{monrat2019survey}. \\
		\hline
		Trasparenza & Tutte le transazioni sono verificabili pubblicamente (nelle \textit{blockchain permissionless}) oppure dai membri autorizzati (nelle \textit{blockchain permissioned}) \cite{crosby2016blockchain}. \\
		\hline
		Sicurezza & L’uso combinato della crittografia asimmetrica per l’autenticazione e dei protocolli di consenso distribuito per l’integrità del registro, rende il sistema resistente ad attacchi e frodi. \\
		\hline
	\end{tabular}
}
\caption{Caratteristiche della tecnologia \textit{blockchain}}
\label{tab:blockchain_vantaggi}
\end{table}


% di questa trasformazione è rappresentato dagli \textbf{\textit{Smart Contract}}, ossia programmi immutabili memorizzati direttamente all'interno del registro distribuito ed eseguiti esclusivamente al verificarsi di determinate condizioni \cite{buterin2013ethereum}. Questi permettdono di trasformare


\subsubsection{Soluzioni esistenti}


La maggior parte degli studi esaminati mostra un ricorso diffuso a \textit{Smart Contract} per la gestione delle donazioni e a meccanismi di approvazione basati sul consenso dei donatori, per l'approvazione delle richieste di spesa.



Gli \textit{Smart Contract} sono programmi immutabili, registrati direttamente sulla \textit{blockchain}, che vengono eseguiti automaticamente al verificarsi di condizioni prestabilite.





%Tuttavia, nonostante i progressi introdotti, permangono alcune limitazioni comuni, quali un coinvolgimento talvolta limitato dei donatori nelle decisioni successive alla raccolta e una tracciabilità non sempre completa dell’intero ciclo di vita della donazione. 
%In particolare, emerge la necessità di modelli che consentano ai donatori di partecipare attivamente alle decisioni relative all’utilizzo delle risorse raccolte al fine di verificare la coerenza tra gli obiettivi dichiarati e l’effettivo impiego dei fondi. Tali aspetti risultano centrali per la costruzione di un rapporto di fiducia duraturo, soprattutto in contesti filantropici.



%cite shelke
%Alcuni ricercatori studiano la possibilità di integrare la blockchain nelle piattaforme esistenti tipo per garantire maggiore trasparenza nella gestione dei fondi oppure per 8min


%https://link.springer.com/chapter/10.1007/978-981-15-1137-0_6#Sec7


%In particolare, la natura decentralizzata della blockchain consente di ridurre il ruolo dell’intermediario fiduciario, mentre l’immutabilità dei dati registrati garantisce l’integrità delle informazioni relative alle transazioni e agli eventi associati alle campagne di raccolta fondi. Ogni operazione può essere verificata pubblicamente, aumentando il livello di trasparenza nei confronti di tutti gli attori coinvolti.


%Risulta evidente che il crowdfunding rappresenti uno strumento fondamentale per la democratizzazione dell’accessibilità ai finanziamenti per una vasta gamma di progetti e organizzazioni; tuttavia, i limiti caratteristici delle piattaforme tradizionali hanno evidenziato la necessità di un’evoluzione verso modelli Web3 – Kickstarter stessa ne è ha preso atto in un comunicato del 2021
