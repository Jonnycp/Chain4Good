% !TeX spellcheck = it_IT
% !TEX TS-program = pdflatex
% !TEX root = ../main.tex


% ********************************************************************
\thispagestyle{plain}

\noindent

%membro 1
\begin{minipage}[t]{0.28\textwidth}
	\centering
	\includegraphics[width=\textwidth]{images/avatar.jpg}
\end{minipage}
\hfill
\begin{minipage}[t]{0.68\textwidth}
	\raisebox{0.6\height}{%
		\parbox{\textwidth}{
			\textbf{Angelica De Feudis}\\
			%Breve descrizione.
	}}
\end{minipage}

%membro2
\begin{minipage}[t]{0.28\textwidth}
	\centering
	\includegraphics[width=\textwidth]{images/avatar.jpg}
\end{minipage}
\hfill
\begin{minipage}[t]{0.68\textwidth}
	\raisebox{0.6\height}{%
		\parbox{\textwidth}{
			\textbf{Jonathan Caputo}\\
			%Breve descrizione.
	}}
\end{minipage}

%membro3

\begin{minipage}[t]{0.28\textwidth}
	\centering
	\includegraphics[width=\textwidth]{images/avatar.jpg}
\end{minipage}
\hfill
\begin{minipage}[t]{0.68\textwidth}
	\raisebox{0.6\height}{%
		\parbox{\textwidth}{
			\textbf{Luca Gentile}\\
			%Breve descrizione.
	}}
\end{minipage}




% IEE.pdf per evitare comportamenti fraudolenti - Qui lo studio introduce un elemento critico: la giustificazione della spesa si basa sulla vigilanza attiva degli utenti. Il documento specifica che i donatori, prima di votare, hanno la responsabilità e il potere di "controllare manualmente sui siti web online" se l'indirizzo del portafoglio del destinatario (il fornitore dichiarato nella richiesta) è legittimo o se è inserito in qualche "blacklist" (lista nera) di truffatori,. In sostanza, la garanzia non è automatica, ma deriva dal controllo collettivo: "mi fido a sbloccare i fondi solo se vedo che l'indirizzo di destinazione corrisponde davvero al fornitore di servizi promesso".
%Se i donatori si accorgono che il beneficiario sta cercando di appropriarsi indebitamente dei fondi (es. inserendo il proprio wallet al posto di quello di un fornitore), hanno a disposizione un'arma di difesa:
%• Possono votare per segnalare la richiesta come fraudolenta ("Report Fraud").
%• Se il 50% o più dei donatori segnala la frode, la richiesta viene respinta.
%• Lo studio prevede una regola rigida: se vengono rilevate due richieste fraudolente (o una singola richiesta che coinvolge l'intero capitale), il progetto viene immediatamente cancellato dal sistema e lo Smart Contract esegue il rimborso automatico di tutti i fondi residui ai donator





