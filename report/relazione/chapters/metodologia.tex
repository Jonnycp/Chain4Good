% !TeX spellcheck = it_IT
% !TEX TS-program = pdflatex
% !TEX root = ../main.tex


% ********************************************************************
\section{Metodologia di progetto}
\label{sec:metodologia}
% ********************************************************************

\subsection{Modello di processo}
Per lo sviluppo di questo sistema è stato adottato un modello di processo \textit{Agile} di tipo \textit{Incrementale}. Questa scelta è motivata dalla necessità di coniugare la flessibilità dei metodi agili, con la capacità del modello incrementale di gestire le fasi di sviluppo in maniera concorrente e sovrapposta. \\
Il coordinamento del \textit{team}, invece, ha seguito la tecnica \textit{Scrum}. In particolare, le riunioni periodiche hanno permesso una gestione dinamica del \textit{product backlog} (elenco delle attività da svolgere) e un monitoraggio costante dello stato di avanzamento del progetto, garantendo un'integrazione continua dei risultati discussi. 

L'orientamento Agile si è manifestato sin dalle fasi iniziali. Le sessioni di \textit{brainstorming} effettuate hanno permesso di proporre e analizzare diverse alternative progettuali. La decisione di abbandonare la proposta iniziale in favore di una più rispondente alle indicazioni dei referenti riflette i principi cardine del Manifesto Agile, quali: collaborazione con gli \textit{stakeholder} e risposta al cambiamento.\\
L'adozione del modello incrementale, invece, ha permesso di ottimizzare i tempi di sviluppo. Il progetto, infatti, non è stato condotto secondo una sequenza rigida di fasi, ma ha previsto lo svolgimento in parallelo di più attività. 


\subsubsection{Organizzazione del team}
Lo sviluppo concorrente ha richiesto la suddivisione delle responsabilità di progetto in macro-aree (\textit{front-end}, \textit{back-end} e documentazione tecnica), favorendo l’avanzamento simultaneo dei diversi incrementi del sistema. Tale ripartizione, tuttavia, non ha comportato una compartimentazione stagna dei compiti. Al contrario, ogni membro del gruppo ha mantenuto una visione olistica del progetto, partecipando attivamente alla risoluzione delle criticità anche al di fuori della propria area di competenza primaria. Tale impostazione ha favorito una dinamica di supporto reciproco e interdisciplinare. 
Il \textit{team} ha, inoltre, operato seguendo il principio della \textit{Collective Ownership}, estendendo a ciascun membro la responsabilità della qualità globale del prodotto.

Complessivamente, l'approccio adottato ha permesso sia di valorizzare i punti di forza di ogni singolo membro che di trasformare le riunioni in opportunità di apprendimento trasversale e di crescita collettiva.
Il successo della metodologia adottata è risultato fortemente legato ai fattori umani, quali competenza tecnica, condivisione degli obiettivi e cooperazione proattiva all'interno del \textit{team}.


\subsection{Pianificazione delle attività}
La pianificazione delle attività è stata articolata in due momenti distinti: inizialmente è stata definita una \gls{wbs} per scomporre gerarchicamente il progetto in macro-attività.
Successivamente, per gestire la sequenzialità e il parallelismo dei \textit{task} individuati, è stato redatto Diagramma di Gantt. \\
Questo approccio ha permesso di confrontare costantemente i tempi effettivi di esecuzione con la durata stimata in fase di pianificazione, garantendo il rispetto delle scadenze.
%wbs
%gannt


\subsection{Analisi dei rischi}

%Rischi di progetto: relativi alla gestione delle attività e dei tempi di sviluppo. In particolare, la variabilità dei requisiti e la limitata esperienza iniziale con alcune tecnologie utilizzate rappresentano potenziali fonti di ritardo.

%Rischio di dominio/prodotto: Un ulteriore rischio individuato riguarda la volatilità del valore delle criptovalute utilizzate per le donazioni. L’adozione di criptovalute ad alta volatilità potrebbe infatti generare una discrepanza significativa tra il valore atteso delle donazioni e il valore effettivamente disponibile al momento dell’utilizzo dei fondi, con un impatto negativo sia sulla pianificazione delle spese da parte degli enti beneficiari sia sulla percezione di affidabilità del sistema da parte dei donatori. riducendo l’esposizione alle fluttuazioni del mercato e garantendo una maggiore stabilità del valore dei fondi raccolti. Per mitigare tale rischio, il sistema è stato progettato per utilizzare stableco. Questa scelta contribuisce a migliorare la prevedibilità economica delle donazioni e a rafforzare la fiducia degli utenti nel corretto utilizzo delle risorse.

%altri rischi sono connessi alle caratteristiche del sistema sviluppato, come la scalabilità, l’usabilità dell’interfaccia e l’affidabilità complessiva del prototipo.

%Per ciascun rischio sono state individuate possibili azioni di mitigazione, quali lo studio preliminare delle tecnologie, l’adozione di soluzioni architetturali consolidate e l’incremento graduale delle funzionalità del sistema.




\subsection{Stima dei costi}
La stima dei costi del progetto è stata effettuata in termini di tempo/persona, tenendo conto della dimensione del sistema, delle tecnologie adottate e dell’esperienza di ciascuno. Considerata la natura prototipale del progetto, è stata adottata una stima qualitativa ispirata ai modelli algoritmici dei costi, come il CoCoMo.




