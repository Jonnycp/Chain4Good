% !TeX spellcheck = it_IT
% !TEX TS-program = pdflatex
% !TEX root = ../main.tex


% ********************************************************************
\section{Metodologia di progetto}
\label{sec:metodologia}
% ********************************************************************

\subsection{Modello di processo}
Per lo sviluppo di questo sistema è stato adottato un modello di processo \textit{Agile} di tipo \textit{Incrementale}. Questa scelta è motivata dalla necessità di coniugare la flessibilità dei metodi agili, con la capacità del modello incrementale di gestire le fasi di sviluppo in maniera concorrente e sovrapposta. \\
Il coordinamento del \textit{team}, invece, ha seguito la tecnica \textit{Scrum}. In particolare, le riunioni periodiche hanno permesso una gestione dinamica del \textit{product backlog} (elenco delle attività da svolgere) e un monitoraggio costante dello stato di avanzamento del progetto, garantendo un'integrazione continua dei risultati discussi. 

L'orientamento Agile si è manifestato sin dalle fasi iniziali. Le sessioni di \textit{brainstorming} effettuate hanno permesso di proporre e analizzare diverse alternative progettuali. La decisione di abbandonare la proposta iniziale in favore di una più rispondente alle indicazioni dei referenti riflette i principi cardine del Manifesto Agile, quali: collaborazione con gli \textit{stakeholder} e risposta al cambiamento.\\
L'adozione del modello incrementale, invece, ha permesso di ottimizzare i tempi di sviluppo. Il progetto, infatti, non è stato condotto secondo una sequenza rigida di fasi, ma ha previsto lo svolgimento in parallelo di più attività. 


\subsubsection{Organizzazione del team}
Lo sviluppo concorrente ha richiesto la suddivisione delle responsabilità di progetto in macro-aree (\textit{front-end}, \textit{back-end} e documentazione tecnica), favorendo l’avanzamento simultaneo dei diversi incrementi del sistema. Tale ripartizione, tuttavia, non ha comportato una compartimentazione stagna dei compiti. Al contrario, ogni membro del gruppo ha mantenuto una visione olistica del progetto, partecipando attivamente alla risoluzione delle criticità anche al di fuori della propria area di competenza primaria. Tale impostazione ha favorito una dinamica di supporto reciproco e interdisciplinare. 
Il \textit{team} ha, inoltre, operato seguendo il principio della \textit{Collective Ownership}, estendendo a ciascun membro la responsabilità della qualità globale del prodotto. \\
Complessivamente, l'approccio adottato ha permesso sia di valorizzare i punti di forza di ogni singolo membro che di trasformare le riunioni in opportunità di apprendimento trasversale e di crescita collettiva.
Il successo della metodologia adottata è risultato fortemente legato ai fattori umani, quali competenza tecnica, condivisione degli obiettivi e cooperazione proattiva all'interno del \textit{team}.


\subsection{Pianificazione delle attività}
La pianificazione delle attività di progetto è stata condotta mediante la definizione di un insieme strutturato delle principali attività da svolgere, riportate in Tabella~\ref{table:pianificazione}. \\
Successivamente, al fine di gestire la sequenzialità e il parallelismo tra i \textit{task} individuati, è stato elaborato un Diagramma di Gantt. \\
Tale approccio ha consentito di organizzare il lavoro in modo strutturato e di definire un riferimento temporale complessivo per l’esecuzione del progetto.


\begin{table}[h]
	\centering
	\small
	\renewcommand{\arraystretch}{1.2}
	\begin{tabularx}{\textwidth}{|
			p{1.2cm}|
			X|
			>{\centering\arraybackslash}p{1.8cm}|
		}
		\hline
		\textbf{ID} & \textbf{Descrizione delle attività} & \textbf{Durata (ore)} \\
		\hline
		\multicolumn{3}{|l|}{\textbf{Fase 1 – Ideazione e analisi}} \\
		\hline
		T1 & \textit{Brainstorming} e definizione dell’idea progettuale & 10 \\
		T2 & Analisi del problema e del dominio applicativo &  5 \\
		T3 & Analisi dei requisiti e studio della fattibilità & 5 \\
		T4 & Analisi dei rischi e criticità & 5 \\
		T5 & Studio delle tecnologie e della \textit{blockchain} & 10 \\
		\hline
		\multicolumn{3}{|l|}{\textbf{Fase 2 – Progettazione}} \\
		\hline
		T6 & Progettazione dell’architettura del sistema & 5 \\
		T7 & \textit{Design} delle interfacce utente & 15 \\
		\hline
		\multicolumn{3}{|l|}{\textbf{Fase 3 – Implementazione del sistema}} \\
		\hline
		T8 & Sviluppo del \textit{Backend} & 40 \\
		T9 & Sviluppo del \textit{Frontend} & 40 \\
		T10 & Redazione della documentazione tecnica & 30 \\
		\hline
		\multicolumn{3}{|l|}{\textbf{Fase 4 – Test e miglioramenti}} \\
		\hline
		T11 & Integrazione dei componenti & 10 \\
		T12 & Verifica del prodotto e delle funzionalità implementate & 5 \\
		T13 & Correzione dei \textit{bug} e perfezionamenti & 6 \\
		\hline
		\multicolumn{3}{|l|}{\textbf{Fase 5 – Conclusione}} \\
		\hline
		T14 & Consegna & 2 \\
		\hline
		\multicolumn{2}{|r|}{\textbf{Totale ore}} & \textbf{228} \\
		\hline
	\end{tabularx}
	\caption{Pianificazione delle attività di progetto.}
	\label{table:pianificazione}
\end{table}

\subsubsection{Stima dei costi}
La stima dei costi di progetto è stata effettuata sulla base della durata stimata delle attività definite in fase di pianificazione, come riportato nella Tabella~\ref{table:pianificazione}. \\
In particolare, l’impegno richiesto è stato valutato operando una scomposizione delle principali fasi progettuali in \textit{task} elementari, associando a ciascuno di essi una stima temporale. Operando in questo modo, il costo complessivo stimato risulta pari a \textbf{228 ore}. \\
Tale valore rappresenta l’\textit{effort} espresso in ore/persona, ottenuto come somma delle ore dedicate da ciascun membro del gruppo alle attività previste.\\
La valutazione ha tenuto conto della natura prototipale del progetto, della dimensione del sistema, delle tecnologie adottate e del livello di esperienza dei membri del gruppo di lavoro.
Tuttavia, trattandosi di un progetto a scopo didattico, la stima dei costi non è stata tradotta in un valore economico monetario, ma è stata utilizzata come strumento di supporto alla pianificazione e alla verifica della fattibilità del progetto entro i vincoli temporali prestabiliti.

\clearpage

% Per fornire una base quantitativa alla stima, si è fatto riferimento al modello algoritmico \gls{cocomo}. Tuttavia, la stima non è stata espressa in termini di \textit{person-month}, come previsto dal modello originale, bensì riformulata in termini di ore/persona, risultando più adeguata al contesto accademico e alla natura non industriale del progetto. Per queste ragioni, la stima dei costi non è stata tradotta in un valore economico monetario, ma utilizzata come strumento di pianificazione e controllo dell’impegno richiesto, al fine di verificare la sostenibilità del progetto rispetto ai vincoli temporali prefissati.

% In particolare, la pianificazione dei \textit{task} riportata in Tabella~\ref{table:pianificazione}, ha consentito di stimare in modo puntuale l’impegno complessivo richiesto per lo sviluppo del progetto.

\subsection{Analisi dei rischi}
L’analisi dei rischi è stata condotta al fine di individuare le principali criticità che avrebbero potuto incidere sullo sviluppo e sul corretto funzionamento della piattaforma. 

In primo luogo, sono stati considerati i rischi di progetto legati la dominio applicativo del \gls{cf}. in particolare, è stata analizzata la possibilità che i fondi raccolti non siano utilizzati per i fini dichiarati dall'Ente promotore. 
Tuttavia, nel sistema proposto, tale rischio viene mitigato attraverso un meccanismo di custodia decentralizzata dei fondi, implementato mediante uno \textit{Smart Contract} che sblocca le risorse esclusivamente a seguito dell'approvazione delle richieste di spesa da parte dei donatori. 

Un ulteriore rischio di progetto è rappresentato dalla volatilità delle criptovalute utilizzate per le donazioni, che può determinare una variazione del controvalore economico dei fondi raccolti rispetto al \textit{budget} inizialmente previsto. Sebbene questo rischio non possa essere eliminato, la registrazione \textit{on-chain} delle transazioni garantisce trasparenza e tracciabilità sull’andamento delle risorse.

Accanto a questi aspetti, sono stati presi in esame i rischi di processo, quali la possibile sottostima dei tempi di sviluppo e l'evoluzione dei requisiti nel corso del progetto. Per ridurre l'impatto di tali incertezze è stato adottato un approccio di sviluppo di tipo incrementale, che ha consentito di adattare progressivamente le soluzioni progettuali sulla base delle evidenze emerse durante l'implementazione. 



% nell'analisi dei rischi di progetto potremmo anche aggiungere il fatto che c'è il rischio che i soldi non vengano utilizzati per i fini dichiarati -> \item Vincolo di sequenzialità sulle richieste di spesa: l'Ente non deve poter sottomettere una nuova richiesta di spesa se non ha preventivamente caricato la prova di acquisto relativa alla richiesta precedentemente approvata;


%Rischi di progetto: relativi alla gestione delle attività e dei tempi di sviluppo. In particolare, la variabilità dei requisiti e la limitata esperienza iniziale con alcune tecnologie utilizzate rappresentano potenziali fonti di ritardo.

%Rischio di dominio: Un ulteriore rischio individuato riguarda la volatilità del valore delle criptovalute utilizzate per le donazioni. L’adozione di criptovalute ad alta volatilità potrebbe infatti generare una discrepanza significativa tra il valore atteso delle donazioni e il valore effettivamente disponibile al momento dell’utilizzo dei fondi. 
%Per mitigare tale rischio, il sistema è stato eprogettato per utilizzare stableco. Questa scelta contribuisce a migliorare la prevedibilità economica delle donazioni e a rafforzare la fiducia degli utenti nel corretto utilizzo delle risorse.
% La soluzione più citata per mitigare l'incertezza finanziaria è l'uso di stablecoin (come USDT o USDC), ovvero criptovalute ancorate a una valuta fiat.

%altri rischi sono connessi alle caratteristiche del sistema sviluppato, come la scalabilità, l’usabilità dell’interfaccia e l’affidabilità complessiva del prototipo.

%Per ciascun rischio sono state individuate possibili azioni di mitigazione, quali lo studio preliminare delle tecnologie, l’adozione di soluzioni architetturali consolidate e l’incremento graduale delle funzionalità del sistema.






