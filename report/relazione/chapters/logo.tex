% !TeX spellcheck = it_IT
% !TEX TS-program = pdflatex
% !TEX root = ../main.tex




\begin{figure}[h]
	\centering
	\vspace*{3cm} 
	\includegraphics[width=0.80\textwidth]{images/Login.pdf}
	\vspace{3cm}
\end{figure}

\newpage







\begin{comment}
	
	
	
	\subsubsection{Requisiti funzionali}
	I requisiti funzionali definiscono le funzionalità che la piattaforma deve implementare. Essi vengono di seguito categorizzati in base agli attori che interagiscono con il sistema. 
	\clearpage
	\begin{enumerate}
		\item l'Ente Beneficiario:
		\begin{itemize} % [start=1,label={RF\arabic*:}] ed enumerate
			\item \textbf{Creazione progetto}: l'Ente deve poter avviare una nuova iniziativa di raccolta fondi definendone nome, \textit{budget target} e data di scadenza.
			\item \textbf{Inserimento di una richiesta di spesa}: l'Ente deve poter richiedere il rilascio di una parte dei fondi raccolti avanzando una richiesta di spesa e allegando il relativo preventivo;
			\item \textbf{Caricamento della prova di acquisto}: l'Ente deve poter caricare la fattura che attesti l'impiego dei fondi precedentemente sbloccati;
			\item \textbf{Vincolo di sequenzialità sulle richieste di spesa}: l'Ente non deve poter sottomettere una nuova richiesta di spesa se non ha preventivamente caricato la prova di acquisto relativa alla richiesta precedentemente approvata;
		\end{itemize}
		
		
		\item per i Donatori:
		\begin{itemize} %[start=5,label={RF\arabic*:}]
			\item \textbf{Visualizzazione dei progetti:} il donatore deve poter visualizzare l'elenco delle iniziative di \gls{cf} attive e i relativi dettagli;
			\item \textbf{Donazione ad un progetto}: il donatore deve poter selezionare un progetto e scegliere arbitrariamente l'importo da donare;
			\item \textbf{Votazione delle richieste di spesa}: il donatore deve poter visualizzare il preventivo di spesa allegato dall'Ente ed esprimere una preferenza (se favorevole o contrario);
			\item \textbf{Visualizzazione del portafoglio}: il donatore deve poter visualizzare il saldo disponibile;
		\end{itemize}
		
		
		\item per il Sistema (logica implementata tramite \textit{Smart Contract}):
		\begin{itemize}%[start=10,label={RF\arabic*:}]
			\item \textbf{Registrazione delle donazioni}: il sistema deve registrare \textit{on-chain} ogni donazione effettuata;
			\item \textbf{Blocco dei fondi}: il sistema deve impedire il trasferimento dei fondi, previo consenso dei donatori;
			\item \textbf{Gestione del processo di votazione}: il sistema deve avviare, gestire e concludere il processo di votazione per ogni richiesta di spesa;
			\item \textbf{Erogazione automatica dei fondi}: il sistema deve trasferire automaticamente i fondi al \textit{wallet} dell'Ente, qualora la richiesta di spesa venga approvata dai donatori;
			\item \textbf{Registrazione delle operazioni}: il sistema deve registrare \textit{on-chain} le richieste di spesa, gli esiti delle votazioni e il trasferimento dei fondi sbloccati;
			
		\end{itemize}
		
	\end{enumerate}
	
\end{comment}



\begin{comment}
	\subsubsection{Requisiti non funzionali}
	Di seguito si riporta l'elenco dei requisiti non funzionali, ossia tutte le caratteristiche che pur non essendo funzionalità, il sistema deve garantire. 
	
	\begin{enumerate}
		
		\item \textbf{Immutabilità}: ogni transazione relativa a donazioni, votazioni e rilascio di fondi deve essere registrata su un registro distribuito in modo permanente e non modificabile;
		
		\item \textbf{Integrità dei dati}: i file pesanti, come preventivi e prove d'acquisto, devono essere memorizzati \textit{off-chain}. Il sistema deve garantire che tali documenti siano riconducibili in modo univoco alle relative operazioni registrate \textit{on-chain}, impedendone la manipolazione;
		
		\item \textbf{Usabilità}: la \textit{Webapp} deve consentire agli utenti di consultare i dati \textit{on-chain}, come lo storico delle donazioni effettuate, attraverso interfacce intuitive;
		
		\item \textbf{Sicurezza}: l'accesso alle funzionalità della piattaforma e alla consultazione dettagliata dei dati deve essere limitato ai soli utenti autenticati;
		
		\item \textbf{Portabilità}: la \textit{Webapp} deve essere fruibile sia da dispositivi \textit{desktop} che \textit{mobile};
		
	\end{enumerate}
\end{comment}



