\documentclass[italian, 12pt, a4paper]{article}

\usepackage{titleps}
\newpagestyle{classica}{%
	\sethead{}{}{\sectiontitle\quad|\quad\thepage}
}
\usepackage{blindtext}
\usepackage{parskip}

\usepackage[utf8]{inputenc}
\usepackage[T1]{fontenc}
\usepackage[protrusion=true,expansion=true]{microtype}
\usepackage[italian]{babel}
\usepackage{hyperref}
\usepackage{csquotes}
\usepackage{amsmath}
\usepackage{amssymb}
\usepackage{amsfonts}
\usepackage[dvipsnames, table]{xcolor}
\usepackage{graphicx}
\usepackage[style=ieee, sorting=none, backend=biber]{biblatex}
\DefineBibliographyStrings{english}{
	references = {Bibliografia},
}
\usepackage{glossaries}
\usepackage{subfig}
\usepackage{listofitems}
\usepackage[noabbrev, capitalise, nameinlink]{cleveref}
\usepackage{multirow}
\usepackage{booktabs}
\usepackage{tabularray}
\usepackage{makecell}
\usepackage{enumitem}
\usepackage{pdflscape}
\usepackage[
	separate-uncertainty=true,
	table-align-uncertainty=true,
]{siunitx}
\usepackage{booktabs}
\usepackage{multirow}
\usepackage[title]{appendix}
\usepackage[euler-digits]{eulervm}
%\usepackage{lmodern}
\usepackage{palatino} % Use the Palatino font by default



%----------------------------------------------------------------------------------------
%	ACRONYMS and BIBLIO
%----------------------------------------------------------------------------------------
% !TeX spellcheck = it_IT
% !TEX TS-program = pdflatex
% !TEX root = ../main.tex


\newacronym{dlt}{DLT}{Distributed Ledger Technologies}

\newacronym{pow}{PoW}{Proof of Work}

\makeglossaries
\addbibresource{biblio.bib} % The filename of the bibliography



%----------------------------------------------------------------------------------------
%	REPORT INFORMATION
%----------------------------------------------------------------------------------------
\title{Chain4Good} % Thesis title, print it elsewhere with \ttitle
\def\teacher{Prof.ssa Marina \textsc{Mongiello}} % Supervisor's name
\author{
	Angelica \textsc{De Feudis} \\
	Johnatan \textsc{Caputo} \\
	Luca \textsc{Gentile}
	} 
\def\subject{Chain4Good}% Subject area
%\def\project{Implementazione di un sistema di crowdfunding decentralizzato}
\def\project{Progettazione e sviluppo di una piattaforma di crowdfunding con tecnologia blockchain}
\def\keywords{
} % Keywords
\def\university{Politecnico di Bari} % University's name
\def\department{Dipartimento di Ingegneria Elettrica e dell'informazione} % Department's name
\def\faculty{Corso di laurea triennale in ingegneria informatica e dell'automazione} % Faculty's name

\makeatletter
\hypersetup{%
	colorlinks=true,
	linktocpage=true,
	pdfstartpage=1,
	pdfstartview=FitV,
	breaklinks=true,
	pdfpagemode=UseNone,
	pageanchor=true,
	pdfpagemode=UseOutlines,
	plainpages=false,
	bookmarksnumbered,
	bookmarksopen=true,
	bookmarksopenlevel=1,
	pdfhighlight=/O,
	urlcolor=webbrown,
    linkcolor=RoyalBlue,
	citecolor=webgreen,%
	pdfcreator={pdfLaTeX},%
	pdfproducer={LaTeX}%,
	pdftoolbar=true,        % show Acrobat’s toolbar?
	pdfmenubar=true,        % show Acrobat’s menu?
	pdffitwindow=true,     % window fit to page when opened
	pdfpagelayout={SinglePage},
	pdftitle={\@title},    % title
	pdfauthor={\@author},     % author
	pdfproducer={\@author},
	pdfkeywords={\keywords},
	pdfsubject={\subject}   % subject
}
\makeatother


%----------------------------------------------------------------------------------------
%	GRAPHICS SETTINGS
%----------------------------------------------------------------------------------------
\graphicspath{{./figures/}}  % dir in which \includegraphics will look for files

\definecolor{webbrown}{RGB}{162, 23, 23}
\definecolor{RoyalBlue}{RGB}{23, 111, 192}
\definecolor{webgreen}{RGB}{0, 128, 0}
\definecolor{halfgray}{RGB}{140, 140, 140}



%----------------------------------------------------------------------------------------
%	MACROS
%----------------------------------------------------------------------------------------
\DeclareMathOperator*{\argmax}{arg\,max}
\DeclareMathOperator*{\argmin}{arg\,min}
\newcommand{\ie}{\textit{i.e.},\space}
\newcommand{\eg}{\textit{e.g.},\space}

\newcommand{\Metric}[2]{\ensuremath{\text{#1}\allowbreak@{#2}}}  % General command for metrics

% Specific commands for each metric (glossary-aware when needed)
\newcommand{\MRR}[1]{\ensuremath{\gls{mrr}\allowbreak@{#1}}}
\newcommand{\NDCG}[1]{\ensuremath{\gls{ndcg}\allowbreak@{#1}}}
\newcommand{\Hit}[1]{\Metric{Hit}{#1}} % no glossary needed
\newcommand{\Precision}[1]{\Metric{Precision}{#1}} % no glossary needed
\newcommand{\Recall}[1]{\Metric{Recall}{#1}} % no glossary needed

\creflabelformat{equation}{#2\textup{#1}#3}  % Equation (X) -> Equation X


\begin{document}

	\pagenumbering{roman} % Use "roman" page numbering style (i, ii, iii, iv...) for the pre-content pages

	\pagestyle{plain} % Default to the plain heading style until the thesis style is called for the body content

	%----------------------------------------------------------------------------------------
	%	TITLE PAGE
	%----------------------------------------------------------------------------------------
	\makeatletter
	\begin{titlepage}
		\setlength{\parskip}{0pt}
		\begin{center}

			\includegraphics[width=35mm]{poliba_logo}
				\vspace{2mm}

			{\scshape\LARGE \university\par}
				\vspace{8mm} 

			{\scshape \department \par}
				\vspace{1mm}

			{\scshape \faculty \par}
				%\vfill
			\vspace{2mm}
			
			% \vspace{2mm}
			% {\rule{0.9\textwidth}{0.8pt} \par} % riga sopra Chain4Good
				\vspace{19mm}

			{\Huge \bfseries \subject \par}
				\vspace{6.5mm} 

			{\itshape\Large \project \par}
				\vfill 

			\begin{minipage}[t]{0.425\textwidth}
				\begin{flushleft}
					\large
					\emph{Candidati:}\\
					\@author\\[2mm]
				\end{flushleft}
			\end{minipage}
			\begin{minipage}[t]{0.425\textwidth}
				\begin{flushright}
					\large
					\emph{Docente:} \\
					\teacher 
				\end{flushright}
			\end{minipage}


            \vspace{13mm} %qui era \vfill
            
            
			{\rule{0.9\textwidth}{0.8pt} \par} %era 1pt
			\vspace{5mm} %qui era 2mm

			{\large Academic Year: 2025/2026 \par}

		\end{center}
	\end{titlepage}
	\makeatother



	%\cleardoublepage
	%------------------------------------------------------------------------------------
	%	LIST OF CONTENTS/FIGURES/TABLES PAGES
	%------------------------------------------------------------------------------------

	\tableofcontents 
	\clearpage
%	\listoffigures % Prints the list of figures
%	\listoftables % Prints the list of tables
	\printglossary[
		title=List of acronyms,
		toctitle=List of acronyms,
		type=\acronymtype,
		nonumberlist,
	]
	\clearpage

	\pagenumbering{arabic} % Begin numeric (1,2,3...) page numbering

	\pagestyle{classica}


	%----------------------------------------------------------------------------------------
	%	CHAPTERS
	%----------------------------------------------------------------------------------------
	% !TeX spellcheck = it_IT
% !TEX TS-program = pdflatex
% !TEX root = ../main.tex


% ********************************************************************
\section{Introduzione}
\label{sec:introduzione}
% ********************************************************************



%subsection{crowdfunding}
%Nata come infrastruttura di supporto per la criptovaluta Bitcoin, la blockchain si è progressivamente evoluta in una tecnologia \textit{general-purpose}, trovando applicazione in un'ampia gamma di contesti. \cite{panarello2018blockchain}.
	\clearpage
	% !TeX spellcheck = it_IT
% !TEX TS-program = pdflatex
% !TEX root = ../main.tex


% ********************************************************************
\section{Background}
\label{sec:background}
% ********************************************************************

\subsection{Blockchain}

Una blockchain è una base di dati distribuita, condivisa e immutabile, mantenuta da una rete di nodi interconnessi secondo un'architettura \gls{p2p}. Dal punto di vista dei sistemi distribuiti, essa si configura come un sistema che garantisce affidabilità e sicurezza replicando i dati e le operazioni su molteplici nodi indipendenti \cite{rodrigues2010peer}.\\
Tale paradigma architetturale è stato introdotto per la prima volta nel 2008 da Satoshi Nakamoto - pseudonimo utilizzato dall'autore (o dal collettivo) dietro il \textit{white paper} di Bitcoin - come infrastruttura di supporto per la nota criptovaluta. \\
Come suggerito dal termine stesso, la blockchain è una catena sequenziale di blocchi, ciascuno contenente un insieme di transazioni validate dalla rete.
Ogni blocco è collegato al precedente attraverso un riferimento crittografico, detto \textit{hash}. Questa caratteristica rende il registro intrinsecamente resistente alla manomissione: qualsiasi tentativo di alterare le informazioni contenute in un blocco comporterebbe, infatti, l'invalidazione a cascata di tutti i blocchi successivi \cite{monrat2019survey}.

\subsubsection{Transazioni}
La transazione rappresenta l’unità fondamentale di informazione all’interno della blockchain. Essa descrive un’operazione richiesta da un utente, come ad esempio il trasferimento di un asset digitale (\textit{token}) o l'invocazione di uno \textit{Smart Contract} \cite{crosby2016blockchain}.
Ogni transazione è creata da un nodo e viene firmata digitalmente utilizzando una coppia di chiavi crittografiche (crittografia asimmetrica). \\
Prima di essere registrata nella blockchain, ogni transazione viene validata dalla rete secondo regole condivise e, una volta confermata, viene inserita in un blocco.

\subsubsection{Blocchi}

Un blocco è una struttura dati che aggrega un insieme di transazioni validate in un determinato intervallo di tempo. Oltre alle transazioni, ogni blocco contiene un \textit{header} con informazioni di gestione fondamentali, tra cui:
\begin{itemize} 
	\item l’\textit{hash} del blocco precedente, che collega crittograficamente i blocchi tra loro formando la catena;  
	\item un \textit{timestamp}, che indica il momento di creazione del blocco; 
	\item il \textit{Merkle Root}, ovvero la radice di un \textit{Merkle Tree}, una struttura ad albero binario che consente di riassumere tutte le transazioni del blocco in un unico \textit{hash};
	\item dati aggiuntivi richiesti dal protocollo di consenso adottato.
	\end{itemize}
	
L’inclusione dell’\textit{hash} del blocco precedente rende la blockchain una struttura immutabile per costruzione: anche una minima modifica a una singola transazione altererebbe la \textit{Merkle Root} e di conseguenza l’\textit{hash} del blocco, invalidando l’intera catena successiva. Questo meccanismo costituisce uno dei principali fattori di sicurezza della blockchain \cite{monrat2019survey}.

\subsubsection{Protocolli di consenso}

Come precedentemente affermato, la blockchain è un sistema intrinsecamente decentralizzato, per cui la validazione delle transazioni non dipende più da un'autorità centrale ma è subordinata all'approvazione collettiva da parte tutti i nodi della rete.
Il processo di validazione di un blocco impone dunque l'implementazione di algoritmi di consenso, come:

\begin{itemize}
	\item \gls{pow}: rappresenta il protocollo di consenso più noto e storicamente rilevante (introdotto da Bitcoin). In questo meccanismo, i nodi della rete, detti \textit{miner}, competono per risolvere un problema crittografico computazionalmente complesso ma facilmente verificabile. 
	Il \textit{miner} che per primo individua una soluzione valida al problema, acquisisce il diritto di creare un nuovo blocco e propagarlo alla rete.
	Tuttavia, l'aggiunta effettiva del blocco avviene solo se questo viene correttamente validato e accettato dalla maggioranza dei nodi. Se queste operazioni vanno a buon fine, il \textit{miner} riceve una ricompensa, generalmente erogata sotto forma di criptovaluta \cite{tschorsch2016bitcoin}.
	
	\item \gls{pos}: in questo protocollo, il processo di validazione è affidato ad un insieme di nodi scelti in funzione della quantità di criptovaluta vincolata come garanzia, in un processo noto come \textit{staking}. I nodi selezionati come \textit{validatori} sono responsabili della creazione dei blocchi e della verifica delle transazioni. \\ La sicurezza della rete è garantita da un sistema di disincentivi economici: qualora un validatore agisca in modo fraudolento o tenti di compromettere il sistema, incorre nella perdita parziale o totale dei fondi posti in \textit{staking}. 
	Una volta raggiunto il consenso, il blocco viene aggiunto immutabilmente alla blockchain, aggiornando lo stato del \textit{ledger} distribuito.
	
\end{itemize}


\subsubsection{Vantaggi della blockchain}
L'adozione della tecnologia blockchain offre vantaggi significativi rispetto ai sistemi centralizzati tradizionali, derivanti dalla sua architettura distribuita e dall'uso della crittografia:

\begin{itemize} 
	\item \textbf{Decentralizzazione}: l’assenza di un’autorità centrale elimina i \textit{single point of failure} \cite{rodrigues2010peer}: 
	\item \textbf{Immutabilità}: ogni transazione è irreversibile, per cui una volta registrata nella blockchain, non può più essere cancellata o modificata, in quanto farlo altererebbe l’intera catena dei blocchi \cite{monrat2019survey}; 
	\item \textbf{Trasparenza}: tutte le transazioni sono verificabili pubblicamente (nelle \textit{blockchain permissionless}) o dai membri autorizzati (nelle \textit{permissioned}) \cite{crosby2016blockchain}; 
	\item \textbf{Sicurezza}: l’uso combinato di crittografia asimmetrica (per l'autenticazione) e protocolli di consenso distribuito (per l'integrità del registro) rende il sistema resistente ad attacchi e frodi;
	\end{itemize}


\subsection{Ethereum Blockchain}

Ethereum è una piattaforma decentralizzata, \textit{open-source} basata su bockchain, proposta da Vitalik Buterin nel 2013 e resa operativa con il primo rilascio stabile nel 2015 \cite{buterin2014next}. 
A differenza delle blockchain nata esclusivamente per il trasferimento di valore, come Bitcoin, Ethereum introduce un modello che consente di implementare logiche computazionali complesse direttamente sulla blockchain. Tale programmabilità è resa possibile attraverso l’utilizzo degli \textit{smart contract}, programmi autonomi e immutabili eseguiti in modo deterministico dalla rete. 

Per supportare l’esecuzione degli \textit{smart contract} e la gestione delle transazioni, Ethereum adotta un modello basato su account, distinguendo tra \textit{\gls{eoa}} e \textit{Contract Accounts}. \\
Gli \gls{eoa}, comunemente identificati come \textit{wallet}, sono controllati direttamente dagli utenti tramite una coppia di chiavi crittografiche: una chiave privata, utilizzata per firmare digitalmente le transazioni, e una chiave pubblica, che funge da indirizzo dell’account sulla rete. \\
Per effettuare una transazione è necessario che il mittente conosca l’indirizzo, ovvero la chiave pubblica, del \textit{wallet} del destinatario e che firmi digitalmente la transazione prima di inviarla alla rete Ethereum con la propria chiave privata, dimostrando così che il richiedente dell’operazione è l’effettivo titolare del \textit{wallet}. \\
I \textit{Contract Accounts}, al contrario, sono account associati a uno \textit{smart contract} e non sono controllati da una chiave privata. Essi non possono avviare transazioni, ma vengono attivati esclusivamente in risposta a chiamate provenienti da un \gls{eoa} o da un altro smart contract. 

La validazione delle transazioni è affidata all’\gls{evm}, una macchina virtuale decentralizzata e Turing-completa eseguita in modo deterministico da tutti i nodi della rete.
%La \gls{evm} ha il compito di eseguire il codice degli \textit{smart contract} in risposta alle transazioni inviate alla rete, garantendo che le stesse istruzioni producano identici risultati su ogni nodo.




%  Ethereum dispone di una criptovaluta nativa, l’Ether, che oltre a poter essere scambiato fra account, è generato dalla piattaforma stessa come ricompensa ai \textit{miner} per il lavoro computazionale svolto. \\


% Ciascuna transazione generata sulla rete Ethereum, sia essa un semplice trasferimento di Ether, o l’invocazione di una funzione di uno smart contract, comporta un costo che è proporzionale alla complessità computazionale, alla banda utilizzata e/o alla quantità di \textit{storage} necessario. Questo meccanismo interno adottato da Ethereum prende il nome di “gas”. 



%\subsubsection{Smart Contract}

%Gli Smart Contract sono programmi memorizzati ed eseguiti sulla blockchain Ethereum, associati a specifici account di tipo \textit{contract}. Essi vengono attivati automaticamente quando ricevono una transazione o un messaggio e possono leggere e modificare il proprio stato interno, trasferire Ether e interagire con altri.
%Gli Smart Contract possono essere scritti utilizzando diversi linguaggi di programmazione, il più utilizzato dei quali è al momento Solidity. Il codice sorgente scritto in Solidity, o in un qualsiasi altro linguaggio ad alto livello fra quelli supportati, deve essere poi compilato per produrre un \textit{bytecode} pronto per essere eseguito dalla \gls{evm}.



\subsubsection{DApps}

Le \gls{dapps} sono applicazioni decentralizzate che operano su una rete \gls{p2p}. 


\subsubsection{DAO}





	\clearpage
	% !TeX spellcheck = it_IT
% !TEX TS-program = pdflatex
% !TEX root = ../main.tex


% ********************************************************************
\section{Metodologia di progetto}
\label{sec:metodologia}
% ********************************************************************

\subsection{Modello di processo}
Per lo sviluppo di questo sistema è stato adottato un modello di processo \textit{Agile} di tipo \textit{Incrementale}. Questa scelta è motivata dalla necessità di coniugare la flessibilità dei metodi agili, con la capacità del modello incrementale di gestire le fasi di sviluppo in maniera concorrente e sovrapposta. \\
Il coordinamento del \textit{team}, invece, ha seguito la tecnica \textit{Scrum}. In particolare, le riunioni periodiche hanno permesso una gestione dinamica del \textit{product backlog} (elenco delle attività da svolgere) e un monitoraggio costante dello stato di avanzamento del progetto, garantendo un'integrazione continua dei risultati discussi. 

L'orientamento Agile si è manifestato sin dalle fasi iniziali. Le sessioni di \textit{brainstorming} effettuate hanno permesso di proporre e analizzare diverse alternative progettuali. La decisione di abbandonare la proposta iniziale in favore di una più rispondente alle indicazioni dei referenti riflette i principi cardine del Manifesto Agile, quali: collaborazione con gli \textit{stakeholder} e risposta al cambiamento.\\
L'adozione del modello incrementale, invece, ha permesso di ottimizzare i tempi di sviluppo. Il progetto, infatti, non è stato condotto secondo una sequenza rigida di fasi, ma ha previsto lo svolgimento in parallelo di più attività. 


\subsubsection{Organizzazione del team}
Lo sviluppo concorrente ha richiesto la suddivisione delle responsabilità di progetto in macro-aree (\textit{front-end}, \textit{back-end} e documentazione tecnica), favorendo l’avanzamento simultaneo dei diversi incrementi del sistema. Tale ripartizione, tuttavia, non ha comportato una compartimentazione stagna dei compiti. Al contrario, ogni membro del gruppo ha mantenuto una visione olistica del progetto, partecipando attivamente alla risoluzione delle criticità anche al di fuori della propria area di competenza primaria. Tale impostazione ha favorito una dinamica di supporto reciproco e interdisciplinare. 
Il \textit{team} ha, inoltre, operato seguendo il principio della \textit{Collective Ownership}, estendendo a ciascun membro la responsabilità della qualità globale del prodotto.

Complessivamente, l'approccio adottato ha permesso sia di valorizzare i punti di forza di ogni singolo membro che di trasformare le riunioni in opportunità di apprendimento trasversale e di crescita collettiva.
Il successo della metodologia adottata è risultato fortemente legato ai fattori umani, quali competenza tecnica, condivisione degli obiettivi e cooperazione proattiva all'interno del \textit{team}.


\subsection{Pianificazione delle attività}
La pianificazione delle attività è stata articolata in due momenti distinti: inizialmente è stata definita una \gls{wbs} per scomporre gerarchicamente il progetto in macro-attività.
Successivamente, per gestire la sequenzialità e il parallelismo dei \textit{task} individuati, è stato redatto Diagramma di Gantt. \\
Questo approccio ha permesso di confrontare costantemente i tempi effettivi di esecuzione con la durata stimata in fase di pianificazione, garantendo il rispetto delle scadenze.
%wbs
%gannt


\subsection{Analisi dei rischi}

%Rischi di progetto: relativi alla gestione delle attività e dei tempi di sviluppo. In particolare, la variabilità dei requisiti e la limitata esperienza iniziale con alcune tecnologie utilizzate rappresentano potenziali fonti di ritardo.

%Rischio di dominio/prodotto: Un ulteriore rischio individuato riguarda la volatilità del valore delle criptovalute utilizzate per le donazioni. L’adozione di criptovalute ad alta volatilità potrebbe infatti generare una discrepanza significativa tra il valore atteso delle donazioni e il valore effettivamente disponibile al momento dell’utilizzo dei fondi, con un impatto negativo sia sulla pianificazione delle spese da parte degli enti beneficiari sia sulla percezione di affidabilità del sistema da parte dei donatori. riducendo l’esposizione alle fluttuazioni del mercato e garantendo una maggiore stabilità del valore dei fondi raccolti. Per mitigare tale rischio, il sistema è stato progettato per utilizzare stableco. Questa scelta contribuisce a migliorare la prevedibilità economica delle donazioni e a rafforzare la fiducia degli utenti nel corretto utilizzo delle risorse.

%altri rischi sono connessi alle caratteristiche del sistema sviluppato, come la scalabilità, l’usabilità dell’interfaccia e l’affidabilità complessiva del prototipo.

%Per ciascun rischio sono state individuate possibili azioni di mitigazione, quali lo studio preliminare delle tecnologie, l’adozione di soluzioni architetturali consolidate e l’incremento graduale delle funzionalità del sistema.




\subsection{Stima dei costi}
La stima dei costi del progetto è stata effettuata in termini di tempo/persona, tenendo conto della dimensione del sistema, delle tecnologie adottate e dell’esperienza di ciascuno. Considerata la natura prototipale del progetto, è stata adottata una stima qualitativa ispirata ai modelli algoritmici dei costi, come il CoCoMo.





	\clearpage
	% !TeX spellcheck = it_IT
% !TEX TS-program = pdflatex
% !TEX root = ../main.tex


% ********************************************************************
\section{Progettazione e implementazione}
\label{sec:progettazione}
% ********************************************************************

\subsection{L'obiettivo di Chain4Good}

\subsection{Analisi dei requisiti}

\subsection{Analisi SWOT}

\subsection{Architettura del Software}
Prima di poter procedere alla progettazione dell’architettura del sistema da realizzare si è resa necessaria l’individuazione delle tecnologie da utilizzare in fase di sviluppo per poter comprendere come queste potessero interagire tra loro e soddisfare tutti i requisiti funzionali e non funzionali emersi dalla precedente fase di analisi.
\subsubsection{Stack architetturale di una dApp}

	\clearpage
	% !TeX spellcheck = en_US
% !TEX TS-program = pdflatex
% !TEX root = ../main.tex


% ********************************************************************
\section{Prototipo}
\label{sec:prototipo}
% ********************************************************************

	\clearpage
	% !TeX spellcheck = it_IT
% !TEX TS-program = pdflatex
% !TEX root = ../main.tex


% ********************************************************************
\section{Validazione e discussione}
\label{sec:validazione}
% ********************************************************************

Gli obiettivi di progetto sono stati raggiunti con successo. L'applicazione web decentralizzata soddisfa i principali requisiti definiti in fase di analisi, in quanto fornisce un sistema in grado di garantire trasparenza, tracciabilità delle donazioni e coinvolgimento attivo degli utenti.\\
Complessivamente, lo sviluppo della piattaforma è stato condotto ponendo particolare attenzione agli aspetti di sicurezza, affidabilità e qualità del \textit{software}. In particolare, sono state adottate misure preventive volte a mitigare le principali vulnerabilità applicative, quali attacchi di tipo \gls{xss}, SQL injection e \textit{replay attack}, con un focus specifico sui meccanismi di autenticazione. 

Particolare attenzione è stata dedicata alle prestazioni e all’esperienza utente. La gestione della \textit{cache} è stata progettata in modo da garantire un accesso efficiente ai dati, prevedendo meccanismi di invalidazione automatica al superamento di determinate soglie temporali, contribuendo così a una maggiore fluidità dell’applicazione. \\
Analogamente, l’adozione corretta di \textit{framework} orientati al \textit{server-side rendering} ha consentito di migliorare la reattività dell’interfaccia. L’interazione con l’utente è stata progettata, inoltre, per gestire esplicitamente tutti gli stati possibili dell’applicazione, inclusi caricamenti, errori e assenza di risultati, fornendo sempre messaggi informativi chiari.

L'attività di \textit{testing} è stata effettuata progressivamente durante lo sviluppo, piuttosto che sul funzionamento dell'intera piattaforma. In particolare, ogni funzionalità è stata testata immediatamente dopo la sua implementazione attraverso test manuali e mediante l’utilizzo di strumenti di sviluppo dedicati, come \textit{Postman} e richieste HTTP da riga di comando, per validare il corretto comportamento delle API \textit{backend}. Inoltre, sono stati eseguiti test automatizzati sugli \textit{Smart Contract} prima della fase di \textit{deploy}, così da individuare eventuali errori logici o comportamenti inattesi in un ambiente controllato.

Infine, il codice è stato organizzato secondo una struttura modulare e ordinata, con una gestione coerente di file, cartelle, al fine di favorire la manutenibilità e l’evoluzione futura del progetto. Sono state inoltre utilizzate le versioni più recenti e stabili delle librerie adottate.
L’intero sistema è stato concepito per essere facilmente containerizzato mediante \textit{Docker} e \textit{docker-compose}, anche se alcune funzionalità risultano volutamente incomplete o ulteriormente ottimizzabili, in linea con la natura prototipale del progetto.


	\clearpage
	% !TeX spellcheck = en_US
% !TEX TS-program = pdflatex
% !TEX root = ../main.tex



% ********************************************************************
\section{Conclusioni e sviluppi futuri}
\label{sec:conclusioni}
% ********************************************************************

	
	
	
	%----------------------------------------------------------------------------------------
	%	PHY
	%----------------------------------------------------------------------------------------
	\clearpage
	
	\printbibliography[heading=bibintoc]


	%----------------------------------------------------------------------------------------
	%	APPENDIX
	%----------------------------------------------------------------------------------------
    \clearpage
    
    %\begin{appendices}
	 %  % !TeX spellcheck = en_US
% !TEX TS-program = pdflatex
% !TEX root = ../main.tex



% ********************************************************************
\section{Appendice}
\label{appendix:search_results}
% ********************************************************************

    %\end{appendices}

\end{document}